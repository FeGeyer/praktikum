\maketitle
\setcounter{page}{1}
\newpage
\pagenumbering{arabic}
\section{Zielsetzung}
  Ziel des Versuchs ist die Lebensdauerbestimmung von Myonen. Diese entstehen
  in der Hochatmosphäre und erreichen aufgrund ihrer relativistischen Geschwindigkeit
  den Erdboden, was einen Nachweis mit Szintillationsdetektoren möglich macht.
\section{Theorie}
  \subsection{Elementarteilchen}
  Im Standartmodell werden zwei Arten Elementarteilchen unterschieden:
  \begin{itemize}
    \item Bosonen
    \item Fermionen.
  \end{itemize}
  Wärend erstere als Austauschteilchen der fundamentalen Wechselwirkungen fungieren,
  stellen letztere die kleinsten aktuell bekannten Bausteine der Materie da.
  Fermionen werden dabei in Quarks und Leptonen unterteilt. Quarks bilden dabei die
  fundamentalen Bausteine der Hadronen (z.B. Baryonen wie Protonen und Neutronen,
  aber auch Mesonen). Sowohl Leptonen als auch Quarks sind in drei Generationen
  zusammengefasst. Die Fermionengenerationen umfassen:
  \begin{itemize}
    \item[I] Elektron ($\symup{e}^-$) und Elektron-Neutrino ($\symup\nu_\symup{e}$)
    \item[II] Myon ($\symup\mu^-$) und Myon-Neutrino ($\symup\nu_\symup\mu$)
    \item[III] Tauon ($\symup\tau^-$) und Tauon-Neutrino ($\symup\nu_\symup\tau$).
  \end{itemize}
  Elektron, Myon und Tauon sind dabei 1-fach negativ geladen und Massebehaftet,
  die entsprechenden Neutrinos sind ungeladen und haben eine im Standartmodell
  verschwindende Masse.\\
  All diesen Teilchen ist ein entsprechendes Antiteilchen zugeordnet. Für die geladenen
  Teilchen wird dies mit einem hochgestellten $+$ anstelle des $-$ gekennzeichnet, die
  Antiteilchen sind 1-fach positiv geladen.
  Antineutrinos erhalten eine Überstreichung. Lediglich das Antiteilchen des Elektrons
  ist als Positron benannt. Das Elektron ist das einzige stabile Lepton.
  \subsection{Kosmische Myonen}
  Myonen entstehen in vielfältigen Prozessen, beispielsweise beim Zerfall von geladenen
  Pionen:
  \begin{equation*}
    \pi^+ \to \mu^+ + \nu_{\mu} \qquad \text{ und } \qquad \pi^- \to \mu^- + \bar{\nu}_{\mu} .
  \end{equation*}
  Diese entstehen beispielsweise in der Hochatmosphäre in sogenannten ausgedehnten
  Luftshowern (EAS). Insbesondere hochrelativistische Protonen können als primäre
  Partikel einen solchen EAS auslösen. Durch Wechselwirkungen mit Luftmolekülen
  entstehen unter anderem Pionen, aus deren Zerfallsprodukten Myonen frei werden.
  Diese Myonen tragen einen Teil der Energie des Primärteilchens und bewegen sich
  daher aufgrund ihrer im Vergleich zum einfallenden Proton geringen Masse ebenfalls mit
  annäherder Lichtgeschwindigkeit. Aufgrund der resultierenden Zeitdilatation
  können sie trotz ihrer geringen Lebensdauer den Erdboden erreichen.
  \subsection{Verhalten von Myonen in Szintillationsdetektoren}
  Myonen können mit einem Szintillationsdetektor nachgewiesen werden. Bei ihrem Durchgang
  durch den Szintillator deponieren sie einen Teil ihrer kinetischen Energie im Szintillatormaterial.
  Dies äußert sich in Form von Anregungszuständen der Moleküle, bei deren Rückkehr in
  den Grundzustand Photonen frei werden. Diese können durch Sekundärelektronenvervielfacher
  (SEV) nachgewiesen werden.\\
  Es sind nun drei Fälle zu unterscheiden:
    \begin{enumerate}
    \item Das Myon hat auf seinem Weg durch die Atmosphäre bereits viel Energie
    verloren und zerfällt im Detektor:\\
    In diesem Fall ist eine Lebensdauerbestimmung möglich. Beim Eintritt in den Detektor
    wird nach dem oben geschilderten Prinzip ein detektierbares Signal erzeugt. Bei
    seinem Weg durch den Detektor zerfällt das Myon in ein Elektron oder Positron\footnote{
    Bei dem angegeben Zerfallskanal handelt es sich lediglich um den mit Abstand häufigsten,
    es existieren weitere.}:
    \begin{align*}
      \mu^+ &\to e^+ + \nu_{e} + \bar{\nu}_{\mu}\\
      \mu^- &\to e^- + \bar{\nu}_{e} +  \nu_{\mu}.
    \end{align*}
    Das entstehende Elektron (oder Positron) ist wiederum in der Lage das Szintillatormaterial anzuregen.
    Durch die dabei entstehenden Photononen wird ein zweites Signal detektierbar
    und aus der Zeitdifferenz lässt sich die Lebensdauer berechnen.
    \item Das Myon durchquert den Detektor ohne zu Zerfallen:\\
    In diesem Fall wird lediglich das erste Signal detektiert, das zweite nicht.
    Dies ist schaltungstechnisch zu berücksichtigen (siehe Kapitel \ref{sec:Aufbau}).
    \item Einfang negativer Myonen durch Szintillatoratome:\\
    Analog zum Elektroneneinfang können negative Myonen mit einer gewissen Wahrscheinlichkeit
    unter Bildung eines myonischen Atoms eingefangen werden. Wie beim durchqueren
    des Detektorn bleibt hier das zweite Signal aus.
    \end{enumerate}
  \subsection{Lebensdauer von Teilchen}
  \label{sec:Lebensdauer}
  Bei Teilchenzerfällen handelt es sich um statistische Prozesse.
  Zuerst wird die Wahrscheinlichkeit $\symup{d}W$,
  dass ein Zerfall im Zeitraum $\symup{d}t$ eintritt, betrachet.
  Unter Annahme von Proportionalität zwischen $\symup{d}W$ und $\symup{d}t$ folgt der Zusammenhang
  \begin{equation*}
    \symup{d}W = \lambda\symup{d}t,
  \end{equation*}
  $\lambda$ stellt hier eine charakteristische Konstante dar. Die Zerfallswahrscheinlichkeit
  ist offensichtlich unabhängig vom individuellen Alter eines Teilchens.
  Die Zerfälle mehrerer Teilchen sollten statistisch unabhängig von einander sein.
  Unter dieser Annahme folgt weiter:
  \begin{equation*}
    \symup{d}N = -N\symup{d}W = - \lambda N \symup{d}t,
  \end{equation*}
  dabei ist $\symup{d}N$ die Zahl der Teilchen, die im Zeitraum $\symup{d}t$ zerfallen sind, wenn
  $N$ Teilchen beobachtet werden. Für große $N$ lässt sich durch integration das exponentielle
  Zerfallsgesetz gewinnen:
  \begin{equation*}
    \frac{N(t)}{N_0} = \exp{(-\lambda t)}.
  \end{equation*}
  Dabei bezeichnet $\lambda$ hier die teilchenspezifische Zerfallskonstante,
  $t$ die Zeit und $N_0$ die zum Zeitpunkt $t=0$ vorhandenen Teilchen.
  In einem Intervall $[t, \symup{d}t]$ lässt sich daraus die Verteilungsfunktion
  bestimmen, der die Lebensdauern der Teilchen folgen:
  \begin{equation*}
    \symup{d}N \left( t \right) = N_0 \cdot \lambda \cdot \symup{exp} \left( - \lambda t \right) \symup{d}t \; .
  \end{equation*}
  Bestimmen des ersten Momentes dieser Verteilung leifert den Erwartungswert für die Lebensdauer:
  \begin{equation}
    <t> = \tau = \frac{1}{\lambda}.
    \label{eq:tau}
  \end{equation}
  \subsection{Statistische Probleme}
  Wäre es möglich, beliebig viele Lebensdauern zu messen, würde die Verteilung
  der Messwerte im Limes $N \to \infty$ gegen die im Kapitel $\ref{sec:Labensdauer}$
  bestimmte Verteilungsfunktion konvergieren. Experimentell ist jedoch immer nur
  eine Stichprobe zugänglich. Hier kommen insbesondere Punktschätzer wie das
  arithmetische Mittel infrage, um den Erwartungswert einer Stichprobe zu bestimmen.
  Wäre es grundsätzlich möglich, jeden Messwert aufzunehmen,
  würde das arithmetische Mittel auch gegen den Erwartungswert konvergieren.
  Diese Bedingung ist hier jedoch nicht erfüllt, das Intervall der theoretisch
  aufzunehmenden Messwerte wird durch den Versuchaufbau begrenzt. Es muss daher
  auf nichtlineare Ausgleichsrechnung mittels der Methode der kleinsten Fehlerquadrate
  zurückgegriffen werden. Eine gute Abschätzung für die Lebensdauer wird also durch
  Regression durch die Messwerte mit der Verteilungsfunktion gewonnen.
\section{Durchführung}
  \subsection{Versuchsaufbau}
  \label{sec:Aufbau}
  \subsubsection{Grundlegende Messmethode:}
  	Ein Blockschaltbild des Versuchsaufbaus ist in \ref{fig:aufbau} dargestellt.
  	Der Szintillationsdetektor besteht dabei aus einem Edelstahlzylinder, an dessen
  	Enden jeweils ein SEV optisch angekoppelt ist. Das Szintillatormedium ist organisch
  	und im Edelstahltank in Toluol gelöst. Die im Szintillator abgegebenen Lichtimpulse besitzen eine
  	Abklingzeit im Bereich von \SI{10}{\nano\second}.\\
  	Bestimmt werden soll nun die Zeit zwischen dem ersten (Eintritt in den
  	Detektor) und dem zweiten Lichtimpuls (Zerfall im Detektor). Dazu wird ein
  	Zeit-Amplituden-Konverter (TAC) genutzt. Der TAC gibt einen Spannungsimpuls ab,
  	dessen Höhe proportional zum zeitlichen Abstand der
  	beiden Signale ist. Zum Bestimmen der Zeitabstände wird eine Stopp-Uhr genutzt.
    Der erste Impuls startet ein Zählwerk, der zweite stoppt es.
    Der am TAC entstehende Impuls wird anschließend in einem Vielkanalanalysator
  	entsprechend seiner Höhe in einem Kanal einordnet und gespeichert.
  	Die Daten werden über einen Rechner ausgelesen.
  	\begin{figure}[p]
    	\centering
    	\includegraphics[width=0.7\textwidth]{Bilder/AufbauB.png}
    	\caption{Blockschaltbild des Versuchsaufbaus \cite{anleitung}.}
    	\label{fig:aufbau}
  	\end{figure}
  \subsubsection{Filtern von nicht-zerfallenden Myonen.}
  Die oben beschriebene Messmethode ist nicht geeignet, um die in Kapitel \ref{sec:Lebensdauer}
  beschriebenen Fälle auszuschließen, in denen das Myon nicht Zerfällt, also nur
  der Eintrittslichtimpuls im Szintillator entsteht. Diese Fälle werden nun
  durch das schaltungstechnische Einbauen einer Suchzeit $T_s$ über eine monostabile
  Kippstufe (auch Univibrator) gefiltert. Die monostabile Kippstufe wird dabei durch
  den vom SEV über eine Koinzidenzschaltung (siehe Kapitel \ref{Rausch}) einlaufenden
  Impuls nach einer Verzögerung angestoßen und in einen instabilen Zustand gehoben.
  Dadurch wird das an den beiden Ausgängen des Univibrators anliegende Signal so
  getauscht, dass ein H-Signal auf das 2. AND-Gatter und ein L-Signal auf das 1. AND-Gatter
  geben wird. Nach Ablauf von $T_s$ werden die Signale wieder zurückgetauscht.
  Das Signal der Koinzidenz wird ebenfalls an das 1. und 2. AND-Gatter gegeben.\\
  Läuft nun der Einfallimpuls in die Schaltung, so liegen am 1. AND-Gatter zwei
  H-Signale (die Verzögerung vor dem Univibrator sorgt dafür, dass die Ausgänge einige
  \si{\nano\second} später umgetauscht werden). Das 1. AND-Gatter schaltet daher durch
  und den TAC erreicht das Start-Signal. Die monostabile Kippstufe schaltet nun um
  und am 2. AND-Gatter liegt ein H-Signal. Läuft in der Zeit $T_s$ nun das Zerfallssignal
  ein, liegen am 2. AND-Gatter 2 H-Signale und das Stopp-Signal für den TAC wird gegeben.
  Passiert dies nicht schaltet die Kippstufe wieder um und die Messung wird verworfen.\\
  Die Suchzeit muss so gewählt werden, dass sie groß gegenüber der Lebensdauer, aber klein
  gegenüber dem zeitlichen Abstand zwischen zwei einfallenden Myonen ist, damit das
  Stopp-Signal nicht durch ein zweites Myon gegeben wird. Dies ist letztlich jedoch nicht
  auszuschließen und durch diese Schaltung auch nicht filterbar. Die Dauer zwischen
  zwei Myonen ist jedoch statistisch Verteilt, wodurch alle Kanäle gleich stark von
  solchen Fehlmessungen betroffen sind. Es ergibt sich eine kontuierliche Untergrundrate $U$,
  die von den Messwerten jedes Kanals abgezogen werden muss.
  \subsubsection{Rauschunterdrückung}
  \label{Rausch}
  Eine weitere Rauschquelle stellen spontane Elektronenemissionen der Photokathoden
  der SEVs dar. Diese führen zu Spannungssignalen, obwohl kein Myon eingefallen ist.
  Die entstehenden Signale sind jedoch meistens kleiner als die von Myonen verursachten.
  Zur Unterdrückung dieser Signale werden zwei Methoden verwendet:
  \begin{enumerate}
    \item Diskriminatoren:\\
    Beiden SEVs sind Diskriminatoren nachgeschaltet. Diese geben nur dann ein Signal
    ab, wenn das einlaufende Signal eine gewisse Schwelle überschreitet. Diese muss
    so gewählt werden, dass echte Signale möglichst nicht gefiltert werden.
    \item Koinzidenzschaltung:\\
    Weiterhin sind gleich zwei SEVs verbaut, deren Signale über eine Koinzidenzschaltung
    abgeglichen werden. Nur wenn von beiden SEVs innerhalb einer Zeit $\symup\Delta t_K$
    ein Signal an den Eingängen der Koinzidenz ankommt, wird ein Signal weitergegeben.
    Da spontane Emissionen jeweils nur einen SEV betreffen und die Wahrscheinlichkeit,
    dass es an beiden SEVs gleichzeitig zu spontaner Emission kommt relativ gering ist,
    stellt dies eine gute Möglichkeit zur Signalfilterung dar. Die Zeit $\symup\Delta t_K$
    ist jedoch so zu wählen, dass Sie sowohl den Lichtweg zwischen den beiden SEVs
    (ca. \SI{4}{\nano\second} für den Fall, dass ein Signal unmittelbar an einem SEV
    entsteht), als auch Unterschiede in den Kabellängen der beiden Leitungen der
    SEVs zur Koinzidenz berücksichtigt. Letztere können durch eine Verzögerungsschaltung
    aufeinander abgeglichen werden.
  \end{enumerate}
  Auch durch diese beiden Möglichkeiten ist eine totale Rauschunterdrückung nicht möglich.
  \subsection{Versuchsdurchführung}
  \label{sec:Durchführung}
  \subsubsection{Aufbau und Justage des Versuchsaufbaus}
  Die Schaltung wird schrittweise aufgebaut und unter Zuhilfenahme eines schnellen Oszillographen
  überprüft und justiert. Es wird mit dem zur Rauschunterdrückung gedachten
  Teil des Aufbaus begonnen:
  \begin{enumerate}
    \item Nach einschalten der Hochspannung sollen an den SEV Ausgängen Impulse
    unterschiedlicher Höhe abfallen. Die ungefähre Länge wird gemessen.
    \item Die Diskriminatoren werden so einjustiert, dass sie Impulse gleicher Länge und
    Höhe liefert. In diesem Schritt muss ein \SI{50}{\ohm} Widerstand parallel zum
    Oszillatoreingang geschaltet werden, da ansonsten Reflexionen durch Fehlanpassungen
    auftreten. Auch hier wird die ungefähre Länge gemessen.
    \item Über ein Zählwerk wird die Zahl der pro Zeitintervall einfallenden Myonen
    gemessen. Die Diskriminatoren werden so eingeregelt, dass sie zwischen 20 und 40
    Myonen pro Sekunde liegt. An beiden Diskriminatoren sollte in etwa die gleiche
    Rate abfallen.
    \item Es wird die Koinzidenzschaltung angeschlossen und der Ausgang auf ein
    Zählwerk gelegt. Die Zählrate wird abhängig von der Verzögerung gemessen, am
    entsprechenden Graphen sollte sich ein "Plateau" bilden. Bei der zum Maximum
    korespondierenden Verzögerung wird die Messung durchgeführt, aus der Halbwertsbreite
    der Kurve lässt sich später die Verzögerungszeit rekonstruieren.
    \item Zuletzt wird die Zählrate vor und hinter der Koinzidenz verglichen. Sollte
    diese annähernd gleich sein, muss die Diskriminatorschwelle gesenkt werden um die
    Myonenrate zu erhöhen. Ansonsten ist die Koinzidenz wirkungslos.
  \end{enumerate}
  Zum weitern Aufbau wird der Teil vor der Koinzidenzschaltung abgeklemmt und ein
  Doppelimpulsgenerator auf den an der Koinzidenz verbleibenden Eingang gelegt.
  Die Dauer zwischen zwei von Doppelimpulsgenerator gegebenen Inpulsen ist einstellbar
  und lässt sich gut zum Überprüfen der Schaltung nutzen.
  \begin{enumerate}
    \item Der Univibrator wird über die Verzögerungsleitung angeschlossen. An den
    Ausgängen der Kippstufe kann nun die Suchzeit gemessen werden. Diese sollte den
    Zeitmessbereich des TAC nur leicht überschreiten.
    \item Die AND-Gatter werden entsprechend des Blockschaltplans eingebaut. Die von
    den AND-Gattern auf die Eingänge des TAC gehenden Signale müssen den selben Abstand
    haben, der zwischen den Impulsen am Doppelimpulsgenerator eingestllt ist.
    \item Der TAC wird überprüft. Die Höhe der am Ausgang abfallenden Signale muss
    dabei proportional zum eingestellten Impulsabstand sein.
    \item Abschließend wird durch Variation der Impulsabstände überprüft, welcher Kanal
    am Vielkanalanalysator welcher Messzeit entspricht.
  \end{enumerate}
  \subsubsection{Messung}
  Die Messung wird begonnen, indem Zählwerk und Vielkanalanalysator gleichzeitig
  gestartet werden. Die Messzeit beträgt zwischen 20 und \SI{30}{\hour}.
  Zum Beenden der Messung werden Zählwerk und Vielkanalanalysator gleichzeitig gestoppt.
  Aufgezeichnet werden die Ergebnisse des Vielkanalanalysators, die Anazahl der detektierten
  Myonen, die Anzahl der Fehlmessungen sowie der Messzeit.

\section{Auswertung}

\subsection{Fehlerrechnung}
  Für die Auswertung werden zwei Arten Punktschätzer genutzt. Es gibt:
  \begin{equation}
    \overline{T}_\symup{arith.} = \frac{1}{n} \sum_{i=1}^{n} T_{i}
    \label{arith}
  \end{equation}
  den arithmetischen Mittelwert, sowie:
  \begin{equation*}
    \overline{T}_\symup{gew.} = \sum_{i=1}^{n} T_{i}  w_{i}
  \end{equation*}
  den mit den Gewichten $w_{i}$ gewichteten Mittelwert. Die Summe der Gewichte ist
  dabei auf $1$ normiert.
  In Fällen, in denen zwei fehlerbehaftete Größen in einer Gleichung zur Bestimmung
  einer anderen Größe Verwendung finden, berechnet sich der Gesamtfehler
  nach der Gaußschen Fehlerfortpflanzung zu:
  \begin{equation*}
      \symup \Delta f(x_1, x_2, ..., x_n) = \sqrt{\left(\frac{\symup df}{\symup dx_1} \symup \Delta
      x_1 \right)^2 +    \left(\frac{\symup df}{\symup dx_2} \symup \Delta
      x_2 \right)^2 + ... + \left(\frac{\symup df}{\symup dx_n} \symup \Delta x_n \right)^2} \ .
    \end{equation*}
    Für die Fehlerrechnung sowie den mathematischen Teil der Auswertung wird auf $\textsc{Python}$ \cite{python}
    zurückgegriffen:\\
    Arithmetische Mittelwerte werden durch die Funktion $\textsc{mean}$ aus dem Paket $\textsc{Numpy}$ \cite{numpy}
    nach \eqref{arith},
    gewichtete Mittelwerte durch manuelles implementieren der jeweiligen Funktion berechnet.
    Fehlerfortpflanzung wird
    durch die Bibliothek $\textsc{uncertainties}$ \cite{uncertainties} automatisiert.
    Regressionen sowie deren Fehler wurden durch die $\textsc{Numpy}$ Funktion $\textsc{polyfit}$
    nach einer $\textsc{least-squares}$-Methode mit Polynomen der an den entsprechenden Stellen
    angegeben Ordnung durchgeführt. Grafiken wurden mit $\textsc{matplotlib}$ \cite{matplotlib}
    erstellt.

\section{Diskusion}

\newpage
\nocite{*}
\printbibliography
