\maketitle
\setcounter{page}{1}
\newpage
\pagenumbering{arabic}
\section{Zielsetzung}
In der Festkörperphysik und Halbleitertechnik wichtige Informationen über die
Bandstruktur von Halbleitern können durch Magneto-Optische Untersuchungen gewonnen
werden. Insbesondere der $\textsc{Faraday}$-Effekt ist hierbei ein wichtiges
Messwerkzeug. Mithilfe dieses Effekts können effektive Massen von Halbleiterelektronen
bestimmt werden, was Ziel dieses Versuchs ist.

\section{Theorie}
\subsection{Effekive Masse}
Zur Beschreibung der Bandstruktur von Kristallen ist in es in vielen Fällen hinreichend,
sich auf die untere Bandkante der Energiebänder zu beschränken (siehe Abbildung [\textbf{??}]).
Der Verlauf der Energiebänder (entspricht der Elektronenenergie) wird mit einer Funktion
$\varepsilon(\vec{k})$ durch den Wellenzahlvektror $\vec{k}$ beschrieben. Diese, bei
Betrachtung des Gesammtverlaufs des Energiebandes in den meisten Fällen äußerst
komplizierte Funktion kann in der Nähe der Bandkante nach den Komponenten von $\vec{k}$
Taylorentwickelt werden. Sinniger Weise legt man die Bandkante dabei in den Nullpunkt
des Koordinatensystems, sodass um Null entwickelt werden kann. Es folgt eine Taylorreihe:
\begin{equation}
  \varepsilon(\vec{k}) = \varepsilon(0) + \frac{1}{2} \sum_{i=1}^3 \left(
  \pdv[2]{\varepsilon}{k_i} \right)_{k=0} k_i^2 + \mathcal{O}(k_i^3).
  \label{T_eq:1}
\end{equation}
Weiterhin gilt die Beziehung
\begin{equation*}
  \varepsilon = \frac{\hbar^2 k^2}{2 m}
\end{equation*}
und ein Vergleich mit der Taylorreihe zeigt, dass die Größen:
\begin{equation*}
  m_i^* \coloneq \hbar^2 \left\{ \left(\pdv[2]{\varepsilon}{k_i} \right)_{k=0} \right\}^{-1}
\end{equation*}
Massen beschreiben. Diese Größen werden effektive Massen genannt und ermöglichen es,
dass Kristallelektronen als freie Elektronen betrachtet werden, wenn ihre Ruhemassen
in der Schrödingergleichung mit eben diesen effektiven Massen ersetzt werden. Das
Potential kann dann in guter Näherung vernachlässigt werden und für mäßige externe
Felder gilt das 2. $\textsc{Newtonsche}$ Axiom.\\
Für perfekte Kristallsymmetrien vereinfacht sich die Taylorentwicklung \eqref{T_eq:1}
dergestalt, dass aus den offensichtlich elliptischen Energieflächen kugelsymmetrische
werden (in diesem Fall gilt $k_x = k_y = k_z$).
Für solche Flächen lässt sich der $\textsc{Faraday}$-Effekt beschreiben.

\subsection{Zirkulare Doppelbrechung bei optisch aktiver Materie}
Beim eintreten eines linear Polarisierten Lichtstrahls in ein optisch aktives Medium
wird die Polarisationsebene des Strahls gedreht. Dies lässt sich dadurch verstehen,
dass eine linear polarisierte Welle zu gleichen Teilen aus einem rechtszirkular und
einem linkszirkular zur Ausbreitungsrichtung (im folgenden $\vec{z}$) polarisierten
Teil besteht. Die beiden Zirkularisationsrichtungen breiten sich nun mit unterschiedlichen
Phasengeschwindigkeiten (bzw. Wellenzahlvektoren) aus und die linear polarisierte
Welle die nach durchgang durch den Kristall wieder austritt ist um den Winkel $\theta$
gedreht (siehe Abbildung [\textbf{??}]). Der Rotationswinkel berechnet sich zu
\begin{align*}
    \theta &=\frac{L}{2} \left(k_{\text{R}}-k_{\text{L}}\right)\\
          &=\frac{L\omega}{2}\left(\frac{1}{v_{\text{ph}_{\text{R}}}}-\frac{1}{v_{\text{ph}_{\text{L}}}}\right)\\
          &=\frac{L\omega}{2\text{c}}\left(n_{\text{R}}-n_{\text{L}}\right),
\end{align*}
wobei $v_{\text{ph}}=\omega k^{-1}$ die Phasengeschwindigkeit und $n=c v_{\text{ph}}^{-1}$
der Brechungsindex für links- bzw. rechtspolarisiertes Licht in einem Kristall der
Länge $L$ ist. Die Lichtfreuqenz der einlaufenden Welle ist durch $\omega$ gegeben.\\
Dieser, als zirkulare Doppelbrechung bezeichnete, Effekt geht auf im Kristallmedium induzierte
Dipole (permantente Dipole können der Frquenz des einfallenden Lichts nicht folgen) zurück.
Die Polarisation $\vec{P} = \varepsilon_0 \chi \vec{E}$ des Kristalls
steigt damit propotional zur dielektrischen Suszeptibilität
$\chi$. Für isotrope Materialien handelt es sich hier um eine skalare Größe, in anisotropen
Medien ist $\chi$ jedoch Richtungsabhängig und muss daher tensoriell definiert werden.
In vielen Fällen handelt es sich hier um eine Diagonalmatrix, das Material ist dann
nicht doppelbrechend (optisch inaktiv). Doppelbrechende Materialien weisen jedoch Nebenachseneinträge
auf, sodass $\chi$ in einfachen Fällen eine Gestallt:
\begin{equation*}
    \underline{\underline{\chi}} =
    \begin{pmatrix}
      \chi_{\text{xx}} & i\chi_{\text{xy}} & 0 \\
      -i \chi_{\text{xy}}& \chi_{\text{xx}} & 0 \\
      0& 0 & \chi_{\text{zz}}
      \end{pmatrix}
\end{equation*}
annimmt. Für den Drehwinkel nach Durchgang folgt dabei nach Taylorentwicklung in
erster Ordnung in guter Näherung:
\begin{equation}
  \theta = \frac{L \omega}{2 c n} \chi_{xy}.
  \label{T_eq:2}
\end{equation}

\subsection{Zirkulare Doppelbrechung bei optisch inaktiver Materie}
Als $\textsc{Faraday}$-Effekt wird nun das Auftreten des oben beschriebenen Effekts
bei optisch inakiver Materie bezeichnet, wenn ein äußeres Magnetfeld angelegt wird.
Aus der Bewegungsgleichung für ein gebundenes Elektron folgt bei Vernächlässigung
von Dämpfungseffekten und Feldeinflüssen durch das Lichtfeld für hohe
Lichtfrquenzen eine Verschiebungspolarisoation $\vec{P} = -N e_0 \vec{r}$.
Es lässt sich nun das in \eqref{T_eq:2} benötigte Tensorelement $\chi_{xy}$ bestimmen
und man erhält in guter Näherung:
\begin{equation*}
  \theta(\lambda)=\frac{2\pi^2 \text{e}_0^3 \text{c}}{\epsilon_0}\frac{1}{m^2
  \lambda^2 \omega{_0}^4}\frac{N B L}{n}.
\end{equation*}
Hierbei wird durch $N$ die Anzahl der Elektronen pro Volumeneinheit, durch $B$
die externe Magentfeldstärke, durch $\lambda$ die Wellenlänge des einfallenden
Lichts und durch $\omega_0$ die meist im nahen infraroten liegende Resonanzfrequenz
der zu erzwungenen Schwingungen fähigen gebundenen Elektronen bezeichnet. Weiter
lässt sich im Limes $\omega_0 \rightarrow 0$ (also für freie Elektronen) ein Ausdruck
\begin{equation}
  \theta_{\text{frei}}(\lambda)=\frac{\text{e}_0^3}{8 \pi^2 \epsilon_0 \text{c}^3}
  \frac{\lambda^2}{m^2}\frac{N L B}{n}
  \label{T_eq:3}
\end{equation}
entwickeln. Dieser kann unter Verwendung der oben eingeführten effektiven Masse
auch für Kristallelektronen verwendet und zur Bestimmung ebendieser genutzt werden.
\section{Durchführung}
\subsection{Versuchsaufbau}
\subsection{Versuchsdurchführung}
\section{Auswertung}
\subsection{Fehlerrechnung}
\section{Diskusion}
\newpage
\nocite{*}
\printbibliography
