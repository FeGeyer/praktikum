\maketitle
\tableofcontents
\newpage

\section{Zielsetzung}
Ziel dieses Versuches ist es, gedämpfte und erzwungene Schwingungen zu untersuchen.
\section{Theorie}
Ein Schwingkreis besteht in seiner einfachsten Form aus einem Kondensator mit der Kapzität
$C$ und einer Spule mit der Induktivität $L$. Die Energie in diesem Schwingkreis oszilliert
zwischen den beiden Energiespeichern und hat als mögliche Maxima ein maximales magnetisches
Feld in der Spule und einen maximal aufgeladenen Kondensator. Falls ein idealer Draht
vorliegt, wird diese $\textbf{ungedämpfte Schwingung}$ für $t \to \infty$ unverändert schwingen.
\subsection{Gedämpfte Schwingung}
\label{sec:gedaempfteSchwingung}
\begin{figure}
  \centering
  \includegraphics[scale=0.6]{gSchwingkreis.png}
  \caption{Schaltbild eines gedämpften Schwingkreises \cite{anleitung}.}
  \label{fig:1}
\end{figure}

Falls nun aber ein endlicher Widerstand $R$ in den Schaltkreis eingebaut wird, siehe \ref{fig:1},
dann wird ein Teil der elektrischen Energie an diesem ohmschen Widerstand in Wärme umgewandelt.
Damit fallen die Amplituden der Spannung und des Stromes mit der Zeit ab und es entwickelt sich
eine $\textbf{gedämpften Schwingung}$. Das Gesetz zwischen der Absinken der Amplitude und der Zeit
lässt sich aus dem zweiten Kirchhoffschen Gesetz, mit den Spannungen aus Abbildung \ref{fig:1},
herleiten
\begin{equation}
    U_R (t) + U_L (t) + U_{\symup{C}} (t) = 0 \, .
    \label{eqn:1}
\end{equation}
Daraus entwickelt man eine lineare homogene Differentialgleichung 2. Ordnung der Form
\begin{equation}
    \ddot{I}(t) + \frac{R}{L} \ddot{I}(t) +
    \frac{1}{LC} \, I(t) = 0 \, ,
    \label{eqn:2}
\end{equation}
welche als Lösung
\begin{equation}
  I'(t) = e^{-2 \pi \mu t} \left(A e^{i 2 \pi \nu t} + B e^{-i 2 \pi \nu t}\right)
  \label{eqn:3}
\end{equation}
mit $A$ und $B$ als beliebige Zahlen aus $\mathbb{C}$ und den Abkürzungen
\begin{align}
        2 \pi \mu &= \frac{R}{2L} \\
        \label{eq:1}
        2 \pi \nu &= \sqrt{\frac{1}{LC} - \frac{R^2}{4L^2}}
\end{align}
besitzt. Für den weiteren Verlauf ist es erforderlich zu ermitteln, ob $\nu$
reell oder imaginär ist. Deshalb wird eine Fallunterscheidung ausgeführt:
\begin{itemize}
  \item $\nu$ ist reell:

    Damit in diesem Fall $I'(t)$ reell wird, muss $A = \overline{B}$ gelten. Mit
    geeigneten Ansätzen erhält man schließlich
    \begin{equation}
        I(t) = A_0 \, e^{-2 \pi \mu t} \, \symup{cos} \left(2 \pi f t + \eta \right)
        \label{eqn:4}
    \end{equation}
    mit $A_0$ und $\eta$ als beliebige Zahlen aus $\mathbb{R}$ und $f$ als Frequenz der
    Schwingung. Gleichung \eqref{eqn:4} stellt eine Schwingungsgleichung für eine
    $\textbf{gedämpfte Schwingung}$ dar, deren Amplitude offensichtlich exponentiell gegen 0 strebt.
    Für die Schwingungsdauer ergibt sich
    \begin{equation}
      T = \frac{1}{f} = \frac{2 \pi}{\sqrt{\frac{1}{LC} - \frac{R^2}{4L^2}}} \, .
      \label{eqn:5}
    \end{equation}
    Die Abnahmegeschwindigkeit steckt im Exponenten der e-Funktion in \eqref{eqn:4},
    nämlich im $\mu$. Daraus lässt sich die Abklingdauer $T_\symup{ex}$ definieren
    \begin{equation}
        T_\symup{ex} = \frac{1}{2 \pi \mu} = \frac{2L}{R} \, .
        \label{eqn:6}
    \end{equation}
  \item $\nu$ ist imaginär:

    Gleichung \eqref{eqn:3} besteht nur noch aus vollständig reellen Exponentialfunktionen,
    sodass \eqref{eqn:4} keinen oszillatorischen Anteil besitzt. Dies nennt man
    aperiodische Dämpfung. Abhängig von $A$ und $B$ strebt $I(t)$ monoton gegen 0 oder
    erreicht noch einen Extremwert. Für das Experiment von Bedeutung ist der Spezialfall
    \begin{equation}
        \frac{1}{LC} = \frac{R_{\symup{ap}}^2}{4L^2} \, ,
        \label{eqn:7}
    \end{equation}
    der $\textbf{aperiodischer Grenzfall}$ heißt und für den $f = 0$ ist. Die Amplitude des Stroms strebt maximal schnell
    gegen 0 und besitzt keinen Überschwinger.
\end{itemize}
\subsection{Erzwungene Schwingung}
\begin{figure}
  \centering
  \includegraphics[scale=0.4]{eSchwingkreis.png}
  \caption{Schaltbild einer erzwungenen Schwingung \cite{alt}.}
  \label{fig:2}
\end{figure}
Nun wird der $RCL$-Schwingkreis aus Kapitel \ref{sec:gedaempfteSchwingung} um eine
Wechselstromquelle $U(t)$ erweitert, wie in Abbildung \ref{fig:2} zu sehen. Diese regt den Schwingkreis
sinusförmig mit einer eigenen Frequenz zusätzlich an. Nach einer gewissen Einschwingzeit wird der Schwingkreis
mit derselben Frequenz wie die Wechselstromquelle schwingen.
Mit
\begin{equation*}
    U(t) = U_0 \, e^{i \omega t}
\end{equation*}
wird die Differentialgleichung \eqref{eqn:2} verändert zu
\begin{equation}
    LC \ddot{U}_{\symup{C}} + RC \ddot{U}_{\symup{C}} + U_{\symup{C}} = U_0 \, e^{i \omega t} \, .
    \label{eqn:8}
\end{equation}
Um zu ermitteln, wie die Amplitude $U_{{\symup{C}}0}$ der Kondensatorspannung mit dem Phasenunterschied
von der Erregerspannung mit der Amplitude $U_0$ und ihrer Frequenz abhängen, nimmt man den Ansatz
\begin{equation*}
  U_{\symup{C}}(\omega, t) = U_{{\symup{C}}0}(\omega) \, e^{i \omega t}
\end{equation*}
und setzt ihn in \eqref{eqn:8} ein (Mit $U_{{\symup{C}}0}$ als beliebige Zahl aus $\mathbb{C}$). Damit erhält man für die Amplitude
\begin{equation}
    U_{{\symup{C}}0} = \frac{U_0 \left(1 -LC \omega^2 - i \omega RC \right)}{\left(1 - LC \omega^2 \right)^2 + \omega^2 R^2 C^2}
    \label{eqn:9}
\end{equation}
und für die Phasenverschiebung $\phi (\omega)$ zwischen $U_{\symup{C}}(t)$ und $U(t)$
\begin{equation}
    \phi (\omega) = \symup{arctan} \left(\frac{-\omega R C}{1 - LC \omega^2}\right) \, .
    \label{eqn:10}
\end{equation}
Für die Frequenzen $\omega_1$ und $\omega_2$ bei denen die Phasenverschiebung genau
$\frac{\pi}{4}$ bzw. $\frac{3\pi}{4}$ beträgt, gilt dann nach \eqref{eqn:10}
\begin{equation}
  \omega_{1,2} = \pm \frac{R}{2L} + \sqrt{\frac{R^2}{4L^2} + \frac{1}{LC}}
  \label{eqn:15}
\end{equation}
Mit \eqref{eqn:9} erhält man für die Kondensatorspannung in Abgängigkeit von $\omega$ die sogennante
Resonanzkurve
\begin{equation}
  U_{\symup{C}}(\omega) = \frac{U_0}{\sqrt{\left(1- LC \omega^2\right)^2 + \omega^2 R^2 C^2}} \, .
  \label{eqn:11}
\end{equation}
An Gleichung \eqref{eqn:11} lässt sich erkennen, dass die Kondensatorspannung für
$\omega \to \infty$ gegen 0 und für $\omega \to 0$ gegen $U_0$ geht. Allerdings
gibt es eine Frequenz, für die die Kondensatorspannung maximiert wird, sodass $U_{\symup{C}} > U_0$ gilt. Dies wird als
Resonanz mit der Resonanzfrequenz
\begin{equation}
  \omega_\symup{res} = \sqrt{\frac{1}{LC} - \frac{R^2}{2L^2}}
  \label{eqn:12}
\end{equation}
bezeichnet. Falls nun die Resonanzfrequenz ungefähr der Frequenz des ungedämpften Schwingkreises
$\omega_0 = \frac{1}{LC}$, d.h. dass in \eqref{eqn:12} $\frac{1}{LC} >> \frac{R^2}{2L^2}$ gilt,
entspricht, so nennt man dies schwache Dämpfung. Für diesen Fall wird $U_{\symup{C}}$ um den Faktor
\begin{equation}
  q = \frac{1}{\omega_0 RC}
  \label{eqn:13}
\end{equation}
größer als $U_0$. \eqref{eqn:13} nennt man auch die $\textbf{Güte q}$ des Schwingkreises.

Eine weitere wichtige Größe ist die Breite der Resonanzkurve aus Formel \eqref{eqn:11}.
Sie wird aus der Differenz der beiden Frequenzen $\omega_+$ und $\omega_-$ gewonnen,
welche sich dadurch auszeichnen, dass $U_{\symup{C}}(\omega_+)$ und $U_{\symup{C}}(\omega_-)$ um den Faktor
$\frac{1}{\sqrt{2}}$ kleiner sind als das Maximum aus \eqref{eqn:13}. Mit der Näherung
\begin{equation*}
  \frac{R^2}{L^2} << \omega_0^2
\end{equation*}
folgt für die Differenz der Frequenzen
\begin{equation}
    \omega_+ - \omega_- \approx \frac{R}{L} \, .
    \label{eqn:14}
\end{equation}

\section{Durchführung}
\subsection{Versuchsaufbau}
\begin{figure}
  \centering
  \includegraphics[scale=0.3]{aufbau.png}
  \caption{Schaltbild des Versuchsaufbaus \cite{anleitung}.}
  \label{fig:3}
\end{figure}
% Wegen dieses Bildes müssen wir nochmal überlegen
\begin{figure}
  \centering
  \includegraphics[scale=0.05]{foto.jpg}
  \caption{Foto des Versuchsaufbaus.}
  \label{fig:4}
\end{figure}
In Abbildung \ref{fig:4} sieht man den Versuchsaufbau im Foto. Der rote Baustein enthält
alle Bestandteile des $RCL$-Schwingkreises plus zwei feste und einen veränderlichen
ohmschen Widerstand. das Gerät auf der linken Seite fungiert als Nadelimpulsgenerator,
kann aber auch die in Kapitel \ref{sec:gedaempfteSchwingung} beschriebene sinusförmige
Spannung liefern. Das rechte Gerät ist ein digitales Oszilloskop, mit dem die Schwingungen
visualisiert werden. Zu diesem Zweck ist der Tastkopf, in Abbildung \ref{fig:3} zu sehen,
in den Schwingkreis eingebaut worden (rechts unten im roten Bauteil steckend).

\subsection{Versuchsdurchführung}
Zuerst werden die benötigten Werte über die Widerstände und die Induktivität und dergleichen
notiert. Dann wird Abbildung \ref{fig:3} aufgebaut, jedoch ohne Frequenzmesser und Verbindung
vom Nadelpulsgenerator zum Oszilloskop und mit dem kleineren der beiden fest eingebauten
ohmschen Widerstände. Dies geschieht, um die Amplitudenabnahme eines
$\textbf{gedämpften Schwingkreises}$ in Abhängigkeit von der Zeit zu ermitteln und daraus
den effektiven Dämpfungswiderstand zu bestimmen. Sodann wird die Spannung gegen die Zeit auf
dem Oszilloskop aufgetragen. Dabei wird darauf geachtet, dass die Amplitude etwa um den Faktor
3 bis 8 abgenommen hat, bevor ein neues Signal des Nadelpulsgenerators eintrifft. Der Schwingungsverlauf
wird mit der Print-Funktion des Oszilloskops festgehalten und ist in Abbildung \ref{fig:5} einsehbar.
\begin{figure}
  \centering
  \includegraphics[scale=0.5]{A.png}
  \caption{Schwingsverlauf einer gedämpften Schwingung.}
  \label{fig:5}
\end{figure}
Aus Abbildung \ref{fig:5} wird mit etwa 10 Messpunkten die Amplitudenabnahme und
daraus der effektiven Dämpfungswiderstand bestimmt.

Der Aufbau für die Bestimmung des Dämpunfswiderstandes $R_{\symup{ap}}$, bei dem
der aperiodische Grenzfall eintritt, wird die gleiche Schaltung wie für den ersten Teil der
Durchführung verwendet, mit dem einzigen Unterschied, dass diesmal kein Festwiderstand,
sondern der veränderliche ohmsche Widerstand Teil des Schwingkreises ist. Zunächst
wird der eben erwähnte veränderliche Widerstand auf sein Maximum eingestellt. Auf dem
Oszilloskop äußert sich das durch die monotone Abnahme der Kondensatorspannung, d. h.
es gibt also keine Schwingung. Alsdann wird der Widerstand schrittweise verringert,
bis ein Überschwinger erkennbar wird. Dies bedeutet, dass der Schwingfall eingetreten ist.
Nun wird der Widerstand wieder erhöht, bis dieser Überschwinger verschwunden ist.
\begin{figure}[t]
  \centering
  \includegraphics[scale=0.5]{B_nah.png}
  \caption{Spannungsverlauf beim aperiodischen Grenzfall auf einem Oszilloskop.}
  \label{fig:6}
\end{figure}
Dies ist in Abbildung \ref{fig:6} zu sehen. Der eingestellte Widerstand wird als
$R_{\symup{ap}}$ notiert.

Um die Frequenzabhängigkeit der Kondensatorspannung zu untersuchen, wird wieder
die Schaltung aus den vorigen Abschnitten aufgebaut, diesmal mit dem größeren der beiden
Festwiderständ und mit einer Sinusspannung statt Nadelpulsen. Da die Frequenzabhängigkeit
der Phase gleichzeitig ausgeführt wird, wird die Sinusspannung ebenfalls auf dem Oszillsokop
dargestellt. Dann werden beide Spannungsverläufe übereinander gelegt und für etwas mehr als
zehn verschiedene Frequenzen $U_0$, sprich die Amplitude der Sinusspannung, $U_{\symup{C}}$ und
die zeitliche Differenz zwischen den beiden Nulldurchgängen der Spannungen aufgenommen.
Die Periodenlänge lässt sich aus der jweils eingestellten Frequenz bestimmen.

\section{Auswertung}
Die Daten der in der Apperatur verwendeten Bauteile sind wie folgt:
\begin{align*}
  L &= \SI{10.11(3)}{\milli\henry}\\
  C &= \SI{2.098(6)}{\nano\farad}\\
  R_1 &= \SI{48.1(1)}{\ohm}\\
  R_2 &= \SI{509.5(5)}{\ohm}
\end{align*}
\subsection{Zeitabhängigkeit der gedämpften Schwingung}
\begin{figure}[h]
  \begin{subfigure}{0.74\textwidth}
  \centering
    \includegraphics[width=\textwidth]{A.png}
    \qquad
  \end{subfigure}
  \begin{subtable}{0.25\textwidth}
  \centering
  \begin{tabular}{c c}
    \toprule
    $U_{\symup{C}}(t_i)$/\si{\deci\volt} & $t_i$/\si{\micro\second}\\
    \midrule
    7.2 & 40 \\
    6.0 & 70 \\
    5.2 & 100 \\
    4.4 & 130 \\
    3.6 & 160 \\
    3.2 & 180 \\
    2.8 & 210 \\
    2.4 & 240 \\
    2.0 & 270 \\
    1.6 & 300 \\
    \bottomrule
    \end{tabular}
    \qquad
  \end{subtable}
  \caption{In der Tabelle sind die aus dem Oszilloskop-Bild entnommene Datenpaare
  eingetragen. Aufgrund des Verstärkungsfaktors x10 des Tastkopfes sind die Werte
  in \si{\deci\volt} gemessen.}
\label{tab:1}
\end{figure}
Oszilloskop-Bild und daraus entnommenen Wertepaare sind in Abbildung \ref{tab:1}
dargestellt. Durch fitten mit einer Funktion:
\begin{equation}
  U(t)=U_0 \cdot \symup{e}^{-2 \pi \mu t}
\end{equation}
in "curve fit" aus dem Python Paket "scipy optimize" ergeben sich folgende Werte für
$U_0$ und $\mu$:
\begin{align*}
  U_0 &= \SI{0.900(8)}{\volt}\\
  \mu &= \SI[per-mode=reciprocal]{894(12)}{\per\second}
\end{align*}
Aus \eqref{eq:1} folgt dann ein Dämpfungswiderstand von $R_{\symup{eff}} = \SI{113.6(15)}{\ohm}$.
Im Vergleich mit dem verwendeten Widerstand $\symup{R}_1$ zeigt sich eine Abweichung
von im Mittel \SI{65.5}{\ohm}. Diese lässt sich durch den Widerstand der anderen
Schaltungselemente (insbesondere der Leitungen) sowie den Innenwiderstand des
Generators erklären. Weiter berechnet sich nach \eqref{eqn:6} eine Abklingzeit von
$T_{\symup{ex}}=\SI{178(2)}{\micro\second}$.
\subsection{Dämpfungswiderstand des aperiodischen Grenzfalls}
Nach \eqref{eqn:7} folgt ein rechnerischer Dämpfungswiderstand von \SI{4.39(1)}{\kilo\ohm},
bei dem der aperiodische Grenzfall eintritt. Gemessen wurde ein Wert von \SI{3.6}{\kilo\ohm}.
Die Abweichung von im Mittel \SI{0.79}{\ohm} lässt sich zum einen wie oben durch
unbetrachtete weitere Widerstände erklären, zum anderen ist ein genaues Einstellen des
Potentiometers nicht möglich. Insbesondere das Oszilloskop erreichte beim Einstellen
die Grenzen seiner Auflösung, sodass es ein gewisses Intervall gibt, in dem der Überschwinger
auf dem Bild des Oszilloskopes gerade verschwunden ist.
\subsection{Frequenzabhängigkeit der Kondensatorspannung}
\begin{table}
  \caption{Messwerte aus den Messungen der Frequenzabhängigkeiten des Schwingkreises.
  Wieder wurde die Kondensatorspannung mit 10-facher Verstärkung gemessen, sodass sie
  in \si{\deci\volt} angegeben wird. Der Wert $\symup{\Delta} t$ gibt den zeitlichen Versatz
  zwischen Kondensator- und Erregerspannung an.}
  \label{tab:2}
  \centering
  \begin{tabular}{c c c c}
    \toprule
    $\nu / \si{\kilo\hertz}$ & $ \symup{\Delta} t / \si{\micro\second}$ & $U_{\symup{C}}$/\si{\deci\volt} & $ U_{\symup{err}} / \si{\volt}$\\
    \midrule
    21 & 1.6 & 11.0 & 76 \\
    25 & 2.4 & 14.0 & 74 \\
    28 & 2.8 & 17.6 & 74 \\
    29 & 3.2 & 19.2 & 72 \\
    30 & 3.4 & 21.2 & 72 \\
    31 & 4.0 & 24.0 & 72 \\
    32 & 5.0 & 26.0 & 70 \\
    33 & 5.8 & 27.6 & 70 \\
    34 & 6.6 & 28.0 & 70 \\
    35 & 7.2 & 27.6 & 70 \\
    36 & 8.4 & 25.6 & 70 \\
    37 & 8.8 & 23.6 & 70 \\
    38 & 9.2 & 20.8 & 72 \\
    39 & 9.4 & 18.0 & 72 \\
    40 & 9.8 & 16.0 & 72 \\
    \bottomrule
    \end{tabular}
\end{table}
\subsection{Frequenzabhängigkeit der Phase zwischen Erreger- und Kondensatorspannung}
\begin{figure}[p]
  \centering
  \begin{subfigure}{0.7\textwidth}
  \centering
    \includegraphics[width=\textwidth]{Phasehalblog.pdf}
    \caption{Halblogarithmische Darstellung}
    \label{sub:1}
  \end{subfigure}\\
  \begin{subtable}{0.7\textwidth}
  \centering
    \includegraphics[width=\textwidth]{Phaselin.pdf}
    \caption{Lineare Darstellung mit eingezeichneten Frequenzen.}
    \label{sub:2}
  \end{subtable}
  \caption{In Abbildung \subref{sub:1} wurden Phase und Frequenz halblogarithmisch gegeneinander
  abgetragen. Die gleichen Werte sind in Abbildung \subref{sub:2} linear dargestellt. Hier wurden
  zusätzlich die Resonazfrequenz sowie die Frequenzen $\nu_1$ und $\nu_2$, also die,
  bei denen eine Phasenverschiebung von $\frac{\pi}{2}$ bzw. $\frac{3 \pi}{2}$ auftritt,
  eingezeichnet.}
\label{abb:2}
\end{figure}
Die gemessenen Werte sind wie oben in Tabelle \ref{tab:2} dargestellt. In Abbildung
\ref{abb:2} ist der Verlauf der Phasenverschiebung in Abhängigkeit von der Frequenz
der Erregerspannung aufgetragen. \ref{sub:1} ist dabei in halblogarithmischer
Darstellung. Es zeigt sich der zu erwartende Verlauf der Phasenverschiebung.
Grafik \ref{sub:2} wurde in linearer Darstellung belassen, um ein Ablesen der Frequenzen
zu erleichtern. Die abgelesenen Werte sind zusammen mit den aus \eqref{eqn:15}
für die Frequenzen $\nu_1$ und $\nu_2$ bei $\frac{\pi}{2}$ bzw. $\frac{3 \pi}{2}$
und \eqref{eqn:12} für die Resonanzfrequenz $ \nu_{\symup{res}}$ in Tabelle \ref{tab:3} dargestellt.
Es ergeben sich äußerst geringe relative Abweichen zwischen Theorie und Experiment.
Die experimentellen Daten liegen jedoch in keinem Fall im Bereich der Fehlertoleranz
der Theoriewerte.
\begin{table}
  \caption{Für $\nu_1$, $\nu_2$ sowie $ \nu_{\symup{res}}$ aus dem Experiment bestimmte
  und errechnete Werte. Zusätzlich ist die relative Abweichung der Werte zueinander
  angegeben.}
  \label{tab:3}
  \centering
  \begin{tabular}{c c c c}
    \toprule
    & experimentell & rechnerisch  & realtive Abweichung / \si{\percent}\\
    \midrule
    $\nu_1 / \si{\kilo\hertz}$ & \num{31} & \num{30.78(6)} & \num{0.65} \\
    $\nu_{\symup{res}} / \si{\kilo\hertz}$ & \num{35} & \num{34.09(7)} & \num{2.57} \\
    $\nu_2 / \si{\kilo\hertz}$ & \num{39} & \num{38.80(8)} & \num{0.51} \\
    \bottomrule
    \end{tabular}
\end{table}

\section{Diskussion}
\newpage
\nocite{*}
\printbibliography
