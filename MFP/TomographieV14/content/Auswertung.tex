\section{Auswertung}
\subsection{Quellmessung}
Zu Beginn wird eine Messung der Quelle durchgeführt, um einen Ausdruck für $I_0$
zu erhalten. Das Energiespektrum ist in Abbildung \ref{fig:1} zu sehen. Der Photopeak
liegt bei \SI{662}{\kilo\eV} und weist 10762 Ereignisse in \SI{300}{\second} auf.

\begin{figure}
  \centering
  \includegraphics[width=\textwidth]{Auswertung/Plots/Quellmessung/quellmessung.pdf}
  \caption{Quellmessung der \ce{^{137}_{55}Cs}-Quelle. Dargestellt ist die Einhüllende.}
  \label{fig:1}
\end{figure}

Damit ergibt sich
\begin{equation}
  I_0 = \SI{35,873}{\becquerel}
  \label{eq:i_0}
\end{equation}
als Eingangsrate.

\subsection{Messung des ersten Würfels}
Für die Messung des ersten Würfels, der nur aus der leeren Aluminiumhülle besteht,
wurde wurden die Projektionen aus Abbildung \ref{abb:2} verwendet. Die Absorptionskoeffizienten
berechnen sich für den Fall ungleicher Varianzen nach
\begin{equation}
  \vec{\mu} = \left(A^{\symup T} \, W \, A \right)^{-1} \cdot \left(A^{\symup T} \, W \, \vec{I} \right) \, ,
  \label{eqn:mu}
\end{equation}
mit $\vec{\mu} = (\mu_1, \mu_2, ..., \mu_9)^{\symup T}$. Die Zahl 9 kommt hierbei
von der Anzahl der Elementarwürfel in der Hülle. In Tabelle \ref{tab:Würfel_1}
sind die Absorptionskoeffizienten für die leere Aluminiumhülle zu sehen.

\begin{table}
  \centering
  \caption{Absorptionskoeffizienten des ersten Würfels.}
  \label{tab:Würfel_1}
  \begin{tabular}{c c}
  \toprule
  k & $\mu_k$ / \SI{e3}{\per\centi\meter} \\
  \midrule
  1 & 25.17 \pm \ 0.76 \\
  2 & 12.60 \pm \ 0.56 \\
  3 & 53.32 \pm \ 0.77 \\
  4 & 38.46 \pm \ 0.57 \\
  5 & 17.89 \pm \ 0.59 \\
  6 & 26.04 \pm \ 0.57 \\
  7 & -8.54 \pm \ 0.76 \\
  8 & 33.31 \pm \ 0.57 \\
  9 & 7.24 \pm \ 0.76 \\
  \bottomrule
  \end{tabular}
\end{table}

Da sich keine unterschiedlichen Elementarwürfel in der Hülle befinden, ist hier
die Angabe eines Mittelwertes sinnvoll. Dieser ergibt sich zu
\begin{equation*}
  \mu_{\symup{mittel}} = \SI{0.02283(22)}{\per\centi\meter} \, .
\end{equation*}

\subsection{Messung des zweiten Würfels}

\subsection{Messung des vierten Würfels}

\section{Diskusion}
