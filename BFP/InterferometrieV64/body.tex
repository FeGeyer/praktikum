\maketitle
\setcounter{page}{1}
\newpage
\pagenumbering{arabic}
\section{Zielsetzung}
\section{Theorie}
\section{Durchführung}
  \subsection{Versuchsaufbau}
  \subsection{Versuchsdurchführung}
\section{Auswertung}

\subsection{Fehlerrechnung}
  Für die Auswertung wird als Punktschätzer der arithmetischen Mittelwert
  \begin{equation}
    \overline{T}_\symup{arith.} = \frac{1}{n} \sum_{i=1}^{n} T_{i}
    \label{arith}
  \end{equation}
  genutzt.
    Für die Fehlerrechnung sowie den mathematischen Teil der Auswertung wird auf $\textsc{Python}$ \cite{python}
    zurückgegriffen:\\
    Arithmetische Mittelwerte werden durch die Funktion $\textsc{mean}$ aus dem Paket $\textsc{Numpy}$ \cite{numpy}
    nach \eqref{arith},
    gewichtete Mittelwerte durch manuelles implementieren der jeweiligen Funktion berechnet.
    Fehlerfortpflanzung wird
    durch die Bibliothek $\textsc{uncertainties}$ \cite{uncertainties} automatisiert.
    Regressionen sowie deren Fehler wurden durch die $\textsc{Numpy}$ Funktion $\textsc{curve-fit}$
    durchgeführt.
    Grafiken wurden mit $\textsc{matplotlib}$ \cite{matplotlib}
    erstellt.

\section{Diskusion}

\newpage
\nocite{*}
\printbibliography
