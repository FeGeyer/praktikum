\maketitle
\tableofcontents
\newpage
\section{Theorie}
\label{sec:theorie}
Dieser Versuch basiert zum Großteil auf Ultraschall. Dieser beginnt ungefähr bei
\SI{20}{\kilo\hertz} und geht bis ca. \SI{1}{\giga\hertz}. Unterhalb, von \SI{16}{\hertz}
bis \SI{20}{\kilo\hertz}, liegt der menschliche Hörbereich. Oberhalb von Ultraschall gibt
es noch den Hyperschall und unterhalb des menschlichen Hörbereichs den Infraschall. Schall
lässt sich darstellen als longitudinale Welle
\begin{equation}
    \rho (x, t) = \rho_0 + v_0 \, Z \, \symup{cos} (\omega t - kx)
    \label{eqn:1}
\end{equation}
mit $Z = c \cdot \rho$ als akustische Impedanz, die abhängig ist von der Dichte
$\rho$ des Materials und der materialabhängigen Schallgeschwindigkeit $c$. In Flüssigkeiten
gilt für die Schallgeschwindigkeit
\begin{equation}
    c_{\symup{Fl}} = \sqrt{\frac{1}{\kappa \cdot \rho}}
    \label{eqn:2}
\end{equation}
mit $\kappa$ als Kompressibilität und $\rho$ als Dichte. Bei Festkörpern können im Gegensatz
zu Gasen und Flüssigkeiten nicht nur Longitudinalwellen, sondern auch Transversalwellen auftreten.
Außerdem fließt statt der Kompressibilität $\kappa^{-1}$ in \eqref{eqn:2} das Elastizitätsmodul $E$ ein
\begin{equation}
    c_{\symup{Fe}} = \sqrt{\frac{E}{\rho}} \, .
    \label{eqn:3}
\end{equation}
Allerdings sind Schallgeschwindigkeiten in Festkörpern richtungsabhängig, sodass
sich die Schallgeschwindigkeit von longitudinalen - und transversalen Wellen unterscheidet.

Durch Absorption verliert die Schallwelle bei der Ausbreitung Energie. Beschreiben
lässt sich dies durch die exponentielle Abnahme der Intensität $I$
\begin{equation}
    I(x) = I_0 \cdot \symup e^{-\alpha x} \, .
    \label{eqn:4}
\end{equation}
Dabei fällt die Intensität nach der Strecke $x$ exponetiell mit dem Absorptionskoeffizienten
$\alpha$ ab. Dieser ist zum Beispiel für Luft sehr groß, sodass in der Medizin ein Ultraschallgel
als Medium zwischen Ultraschallsender und Material genutzt wird.

Außerdem wird ein Teil der Schalwelle reflektiert, sobald diese auf eine Grenzfläche trifft.
Das Verhältnis aus reflektierter und einfallender Intensität wird als Reflexionskoeffizient
$R$ bezeichnet
\begin{equation}
  R = \left(\frac{Z_1 - Z_2}{Z_1 + Z_2} \right)^2 \, .
  \label{eqn:5}
\end{equation}
Das $Z$ steht für die bereits erwähnte akustische Impedanz. Aus $T = 1 - R$ lässt
sich der transmittierte Anteil $T$ bestimmen.

Erzeugt werden kann Ultraschall unter Zuhilfenahme des piezo-elektrischen Effekts.
Ein piezoelektrischer Kristall, meist ein Quartz, wird duch ein sich periodisch änderndes elektrisches Feld
zu Schwingungen angeregt und strahlt dadurch Ultraschallwellen ab. Stimmt die Frequenz des
elektrischen Wechselfeldes mit der Eigenfrequenz des Kristalls überein (liegt also ein
Resonanzfall vor),
erreichen die abgestrahlten Ultraschallwellen hohe Amplituden und damit auch hohe Energieichten.
Der piezo-elektrische Effekt lässt sich umkehren und somit lassen sich die
Kristalle auch als Schallempfänger nutzen. Einlaufende Schallwellen regen den Kristall dabei zu Schwingungen an.

Häufig werden in der Medizin Laufzeitmessungen mittels Ultraschall durchgeführt,
um an Informationen zu gelangen. Dabei gibt es zwei Verfahren:
\begin{itemize}
  \item \textbf{Durchschallungsverfahren}
  Bei diesem Verfahren befinden sich Sender und Empfänger gegenüber voneinander.
  Dazwischen befindet sich das zu durchschallende Material. Bei einer Fehlstelle
  in der Probe wird die Intensität abgeschwächt. Im Gegensatz zum Impuls-Echo-Verfahren
  ist keine quantitative Aussage darüber möglich, vgl. Abbildungen \ref{fig:1} und \ref{fig:2}.
  \begin{figure}{h}
    \centering
    \includegraphics[scale=0.5]{durchschall.png}
    \caption{Das Prinzip des Durchschallungsverfahren.}
    \label{fig:1}
  \end{figure}

  \item \textbf{Impuls-Echo-Verfahren}
  Im Gegensatz zum Durchschallungsverfahren fungiert hier der Sender auch als Empfänger.
  Dabei kommt zum Tragen, dass Schalwellen an Grenzwellen teilweise reflektiert werden (siehe oben).
  Fehlstellen können mit diesem Verfahren quantitativ bestimmt werden, so kann man aus der
  Höhe des Echos auf die Größe der Fehlstelle schließen. Außerdem kann mithilfe der Formel
  \begin{equation}
    s = \frac{1}{2} c \, t
    \label{eqn:6}
  \end{equation}
  bei bekannter Schallgeschwindigkeit $c$ die Position $S$ der Fehlstelle bestimmt werden, siehe Abbildung \ref{fig:2}.
  Für die Darstellung der Laufzeitmessung gibt es den A-Scan, B-Scan und TM-Scan.
  \begin{figure}
    \centering
    \includegraphics[scale=0.5]{impuls.png}
    \caption{Skizze des Impuls-Echo-Verfahrens.}
    \label{fig:2}
  \end{figure}
\end{itemize}
\section{Durchführung}
\subsection{Versuchsaufbau}
Das Experiment setzt sich aus Ultraschallechoskop, einer Ultraschallsonde mit
\SI{2}{\mega\hertz} und einem Rechner mit dem Programm EchoView, um die Daten darstellen zu können.
Am Echoskop lässt einstellen, ob eine (Impuls-Echo-Verfahren) oder zwei (Durchschallungsverfahren, siehe Kapitel \ref{sec:theorie})
Ultraschallsonden verwendet werden. Außerdem lässt sich ein Gain in \si{\decibel} und
der TGC (auch in \si{\decibel}) wählen. Als Untersuchungsmaterial stehen verschieden lange Acrylzylinder und
- platten, ein Augenmodell und als Kontaktmittel bidestilliertes Wasser zur Verfügung.
\subsection{Versuchsdurchführung}
Zuerst werden einige Testmessungen durchgeführt und probeweise die Schallgeschwindigkeit
bestimmt. Diese Messungen werden aber im Folgenden nicht weiter behandelt. Danach wird
das Impuls-Echo-Verfahren angewandt. Zu diesem Zweck wurden alle Acrylzylinder mit einer
Schieblehre mit einer Genauigkeit von \SI{0.2}{\milli\meter} vermessen. Als nächstes
wird der Zylinder mit bidestilliertem Wasser behandelt und die Ultraschallsonde angebracht.
Mittels eines A-Scans und Makern werden Amplitude der Reflexion und Laufzeit des Impulses
festgehalten und notiert. Außerdem wird eine eventuell eingestellte Verstärkung (TGC) aufgeschrieben.
Dies wird für ingesamt sieben verschiedene Längen durchgeführt.

Als nächstes wird das Durchschallungsverfahren angewandt. Dafür wird eine zweite Sonde
angeschlossen, die als Empfänger fungiert. die Acrylzylinder werden der Länge nach in
eine Aparatur eingespannnt, die an beiden Enden jeweils eine Sonde hat. Dann werden wieder
Amplitude, Laufzeit und TGC mittels A-Scan für sieben Längen aufgenommen.

Danach wird der ca. \SI{40}{\milli\meter} Acrylzylinder auf zwei Acrylplatten gestellt,
um mithilfe des Cepstrums Mehrfachreflexionen zu analysieren. Die verschiedenen Materialeinheiten
werde mit bidestilliertem Wasser gekoppelt und dann mithilfe des Impuls-Echo-Verfahrens die
Reflexionen aufgenommen. Dabei ist zu beachten, dass die Verstärkung so aufgenommen wird,
dass drei Reflexionen zu sehen sind. Im Programm EchoView lässt sich dann das Cepstrum
als Diagramm auftragen.

Als letztes wird das Augenmodell mit dem Impuls-Echo-Verfahren untersucht. Dazu wird
bidestilliertes Wasser auf die Hornhaut gegeben und dann mit der Ultraschallsonde
die Reflexionen von Iris, Linse und Retina untersucht, siehe Abbildung \ref{fig:3}.
Zu diesem Zweck werden die verschiedene Laufzeiten der Reflexionen aufgenommen und
später mit den realen Längen eines menschlichen Auges abgeglichen.
\begin{figure}
  \centering
  \includegraphics[scale=0.5]{auge.png}
  \caption{Augenmodell im Querschnitt.}
  \label{fig:3}
\end{figure}
\section{Auswertung}
\subsection{Bestimmung der Schallgeschwindigkeit in Acryl}
Die aus der Messung nach dem Impuls-Echo-Verfahren gewonnenen Messwerte finden sich
in Tabelle \ref{tab:1} , die aus dem Durchschallungsverfahren in Tabelle \ref{tab:2}.
\begin{figure}
  \centering
  \begin{subtable}{0.49\textwidth}
    \centering
    \begin{tabular}{c c c c}
      \toprule
      $l$/\si{\milli\metre} & $U$/\si{\volt} & $\symup{\Delta}t$/\si{\micro\second} & TGC/\si{\decibel} \\
      \midrule
      31.00 & 0.202 & 23.16 & 0 \\
      40.10 & 0.193 & 29.68 & 0 \\
      61.58 & 0.310 & 45.62 & 17.81 \\
      71.30 & - & 52.46 & 20.58 \\
      80.20 & 0.154 & 59.66 & 18.16 \\
      102.00 & 0.105 & 75.90 & 32.85 \\
      121.18 & 0.105 & 88.38 & 32.85 \\
      \bottomrule
    \end{tabular}
    \caption{Messwerte der Messung per Impuls-Echo-Verfahren. Bei den $\symup{\Delta}t$-
    Werten handelt es sich um die doppelte Laufzeit.}
    \label{tab:1}
  \end{subtable}
  \begin{subtable}{0.49\textwidth}
    \centering
    \begin{tabular}{c c c c}
      \toprule
      $l$/\si{\milli\metre} & $U$/\si{\volt} & $\symup{\Delta}t$/\si{\micro\second} & TGC/\si{\decibel} \\
      \midrule
      31.00 & 0.729 & 12.48 & 0     \\
      40.10 & 0.759 & 15.46 & 0     \\
      61.58 & 0.271 & 23.70 & 0     \\
      71.30 & -     & 27.18 & 0     \\
      80.20  & 0.154 & 30.50 & 0     \\
      102.00    & 0.271 & 38.71 & 15.48 \\
      121.18 & 0.329 & 45.19 & 17.99 \\
      \bottomrule
    \end{tabular}
    \caption{Messwerte der Messung per Durchschallungsverfahren.\\}
    \label{tab:2}
  \end{subtable}
  \caption{$l$ bezeichnet jeweils die Höhe der Zylinder, $U$ die Spannungsamplitude des
  gemessenen Peaks, $\symup{\Delta}t$ den zeitlichen Abstand zwischen senden des Schallimpulses
  und empfangen des Signals und TGC gibt den verwendeten Verstärkungsfaktor für
  den Amplitudenwert an. Die Längenmessung ist auf einen Fehler von $\pm \, \num{0.5}$
  genau.}
\end{figure}
Die Bestimmung der Schallgeschwindigkeiten gescheiht nun durch lineare Regression.
Zu beachten ist, dass sich die Laufzeiten bei der Messung per Impuls-Echo-Verfahren
Verfahrensbedingt auf die doppelte Länge beziehen. Es werden daher in den Rechnungen die halben gemessenen
Laufzeiten betrachtet.
Dies ist notwendig, da innerhalb des Sondenmaterials eine gewisse Strecke zurückgelegt werden
muss, die ansonsten als systematischer Fehler in die Rechnung eingehen würden.
Regression der Länge-Laufzeit Wertepaare mit
\begin{equation}
  t(l) = l \cdot \frac{1}{c} + t_0
\end{equation}
(wobei $c$ die Schallgeschwindigkeit im Zylinder und $t_0$ die Laufzeit innerhalb der
Sonde ist) liefert für das Impuls-Echo-Verfahren:
\begin{align*}
  c &= \SI[per-mode=reciprocal]{2740(24)}{\metre\per\second}\\
  t_0 &= \SI{0.28(25)}{\micro\second}
\end{align*}
und für das Durchschallungsverfahren:
\begin{align*}
  c &= \SI[per-mode=reciprocal]{2727(21)}{\metre\per\second}\\
  t_0 &= \SI{1.0(2)}{\micro\second}
\end{align*}
Messwerte und Regression sind in Abbildung \ref{abb:1} für das Impuls-Echo-Verfahren,
sowie in Abbildung \ref{abb:2} für das Durchschallverfahren dargestellt.
\begin{figure}
  \centering
  \begin{subfigure}{0.49\textwidth}
    \centering
    \includegraphics[width=\textwidth]{Imp.pdf}
    \caption{Impuls-Echo-Verfahren}
    \label{abb:1}
  \end{subfigure}
  \begin{subfigure}{0.49\textwidth}
    \centering
    \includegraphics[width=\textwidth]{Dur.pdf}
    \caption{Durchschallverfahren}
    \label{abb:2}
  \end{subfigure}
  \caption{Dargestellt sind die gemessenen Laufzeiten bei beiden Messmethoden mit Regression.
  Insbesondere in \subref{abb:2} erkennt man die Verschiebung des Graphen auf der y-Achse,
  verursacht durch die Schalllaufzeit innerhalb der Sonde, deutlich.}
\end{figure}

\subsection{Betrachtung des Dämpfungsverhaltens von Acryl}
Nun werden die Dämpfungseigenschaften von Acryl betrachtet. Dabei wird die gemessene
Spannungsamplitude des Signals in Abhängigkeit der zurückgelegten Wegstrecke betrachtet.
Zu beachten sind hier drei Dinge:
\begin{enumerate}
  \item Beim Impuls-Echo-Verfahren wird Verfahrensbedingt die dopplete Strecke zurückgelegt.
  \item Manche Amplituden mussten mit Verstärkung gemessen werden. Diese ist in den Tabellen
  \ref{tab:1} und \ref{tab:2} als "TGC"-Wert in \si{\decibel} angegeben. Die Umrechnung in einen
  linearen Faktor geschieht durch:
  \begin{equation}
    G(g) = 10^{ \left( g \cdot \frac{1}{20} \right)}
    \label{eqqA:1}
  \end{equation}
  mit "TGC"-Wert $g$.
  \item Beim Durchschallverfahren wurde eine Empfangsverstärkung von \SI{10}{\decibel}
  zugeschaltet. Diese kann auch nach \ref{eqqA:1} eliminiert werden. Die für beide Verfahren
  genutzte Sendeverstärkung von ebenfalls \SI{10}{\decibel} verbleibt in den Messwerten,
  da sie die gesuchte Größe $\alpha$ sowie die Vergleichbarkeit beider Verfahren nicht beeinflusst.
\end{enumerate}
Die Dämpfung verläuft exponentiell, der Dämpfungskoeffizient $\alpha$ wird daher durch Regression
mit einer Funktion:
\begin{equation}
  U(l) = U_0 \cdot e^{\alpha l}
\end{equation}
in "phython-scipy" bestimmt. Zu erwähnen ist, dass die für die Zylinder mit Länge
$l=\SI{71.3}{\milli\metre}$ gemessenen Werte nicht genutzt werden können, da dieser
Zylinder aus zwei kürzeren Stücken zusammengesetzt wurde und sich so eine Signalabschwächende
Grenzschicht zwischen den beiden Zylindern bildet. Für das Impuls-Echo-Verfahren ergeben sich:
\begin{align*}
  U_0 &= \SI{0.84(33)}{\volt}\\
  \alpha &= \num{-21(5)}
\end{align*}
und für das Durchschallungsverfahren:
\begin{align*}
  U_0 &= \SI{0.69(18)}{\volt}\\
  \alpha &= \num{-32(11)}
\end{align*}
Die Verläufe mit Regression sind in den Abbildungen \ref{abb:3} und \ref{abb:4} dargestellt.
\begin{figure}
  \centering
  \begin{subfigure}{0.49\textwidth}
    \centering
    \includegraphics[width=\textwidth]{ImpDump.pdf}
    \caption{Impuls-Echo-Verfahren}
    \label{abb:3}
  \end{subfigure}
  \begin{subfigure}{0.49\textwidth}
    \centering
    \includegraphics[width=\textwidth]{DurDump.pdf}
    \caption{Durchschallverfahren}
    \label{abb:4}
  \end{subfigure}
  \caption{Dargestellt sind die gemessenen Signalamplituden in Abhängigkeit
  der Signallaufzeit bei beiden Messmethoden mit Regression.}
\end{figure}
\subsection{Vermessung des Augenmodells}
Die gewonnen Messwerte sind in Tabelle \ref{tab:3} dargestellt. Für die Schallgeschwindigkeit
in der Glaskörperflüssigkeit gilt $c_{GK} = \SI[per-mode=reciprocal]{1410}{\metre\per\second}$
und für die Linse $c_L = \SI[per-mode=reciprocal]{2500}{\metre\per\second}$
\begin{figure}
\begin{subtable}{0.39\textwidth}
  \centering
  \begin{tabular}{c c}
    \toprule
    Grenzschicht & $\symup{\Delta}t$/\si{\micro\second} \\
    \midrule
    Iris & 5.82 \\
    Linse ein & 10.68 \\
    Linse aus & 16.50 \\
    Retina & 69.60 \\
    \bottomrule
  \end{tabular}
  \caption{Ergebnisse der Vermessung des Augenmodells im Maßstab 3:1. Angegeben sind
  die verfahrensbedingt erhaltenen doppelten Laufzeiten.}
  \label{tab:3}
\end{subtable}
\begin{subtable}{0.59\textwidth}
  \centering
  \begin{tabular}{c c c}
    \toprule
    Grenzschicht & $l$/\si{\milli\metre} & $\frac{l}{3}$/\si{\milli\metre} \\
    \midrule
    Iris & 4.11 & 1.37\\
    Linse ein & 7.53 & 2.52\\
    Linse aus & 14.81 & 4.94\\
    Retina & 52.24 & 17.42\\
    \bottomrule
  \end{tabular}
  \caption{Berechnete Abstände zwischen Schallsonde und Grenzschichten, angegeben
  für das Augenmodell (Maßstab 3:1) und zurückgerechnet auf ein menschliches Auge.}
  \label{tab:3}
\end{subtable}
\end{figure}
\section{Diskussion}
\subsection{Schallgeschwindigkeitsmessung}
Bei allen Messwerten muss mit einem nicht quantifizierbaren Fehler gerechnet werden,
da das Ablesen der Werte in der Messsoftware durch das Setzen von Markern geschieht,
die den Peak nie genau treffen können.\\
Beide Ergebnisse liegen innerhalb der gegenseitigen Messungenauigkeit. Ebenfalls liegt
der Literaturwert von \SI[per-mode=reciprocal]{2730}{\metre\per\second} in den Intervallen
beider Ergebnisse. Das Fehlerintervall des durch die Durchschallungsmessung gewonnen Wertes
ist jedoch geringer als das des Wertes aus der Impuls-Echo-Methode. Dies erscheint logisch,
muss das Signal bei der Impuls-Echo-Methode den doppelten Weg zurücklegen, weshalb eventuelle
Fehler im Material einen größeren Einfluss haben und somit eine größere systematische
Fehlerquelle bieten. Auffällig ist die große Abweichung zwischen den Werten der sondeninternen
Laufzeit. Da der Fehler des bei der Messung per Impuls-Echo-Verfahren gewonnenen Wertes
jedoch annähernd so groß ist wie der Wert selbst liegt die Vermutung nah, dass hier ein
Problem vorliegt. Da jedoch beide Werte aus einer Ausgleichsrechnung mit einem sehr
begrenzten Datensatz gewonnen wurden und keine Literaturwerte vorhanden sind, kann hier nur
vermutet werden. Das aufnehmen weiterer Messwerte würde helfen, die Ergebnisse zu verifizieren.
\subsection{Dämpfungsverhalten}
Die erhaltenen Werte liegen jeweils im Bereich der gegenseitigen Messungenauigkeit.
Hohe relative Fehler der Werte lassen sich wieder durch den begrenzten Datensatz
erklären. Ebenfalls problematisch sind die langen Laufzeiten bei der Impuls-Echo-Messung.
Die erhaltenen Daten sind daher bereits stärker gedämpft, weshalb die Ausgleichsrechnung
statisch signifikantere Fehler ergeben sollte.

\newpage
\nocite{*}
\printbibliography
