\maketitle
\setcounter{page}{1}
\newpage
\pagenumbering{arabic}
\section{Zielsetzung}
In der Festkörperphysik und Halbleitertechnik wichtige Informationen über die
Bandstruktur von Halbleitern können durch Magneto-Optische Untersuchungen gewonnen
werden. Insbesondere der $\textsc{Faraday}$-Effekt ist hierbei ein wichtiges
Messwerkzeug. Mithilfe dieses Effekts können effektive Massen von Halbleiterelektronen
bestimmt werden. Dies ist Ziel des Versuchs.
\section{Theorie}
\subsection{Effkive Masse}
Zur Beschreibung der Bandstruktur von Kristallen ist in es in vielen Fällen hinreichend,
sich auf die untere Bandkante der Energiebänder zu beschränken (siehe Abbildung \ref{T_Abb:1}).
Der Verlauf der Energiebänder (entspricht der Elektronenenergie) wird durch eine Funktion
$\varepsilon(\vec{k})$ durch den Wellenzahlvektro $\vac{k}$ beschrieben. Diese, bei
Betrachtung des Gesammtverlaufs des Energiebandes in den meisten Fällen äußerst
komplizierte Funktion, kann in der Nähe der Bandkante nach den Komponenten von $\vec{k}$
Taylorentwickelt werden. Sinniger Weise legt man die Bandkante dabei in den Nullpunkt,
sodass um Null entwickelt werden kann.
\section{Durchführung}
\subsection{Versuchsaufbau}
\subsection{Versuchsdurchführung}
\section{Auswertung}
\subsection{Fehlerrechnung}
\section{Diskusion}
\newpage
\nocite{*}
\printbibliography
