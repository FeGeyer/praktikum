\maketitle
\setcounter{page}{1}
\tableofcontents
\newpage
\pagenumbering{arabic}
\section{Theorie}
\begin{equation}
  \symup{\Delta} d = z \frac{\lambda}{2}
  \label{eq:1}
\end{equation}
\begin{equation}
  b \cdot \symup{\Delta}n = \frac{z \lambda}{2}
  \label{eq:2}
\end{equation}
\begin{equation}
  n(p_0, T_0) = 1 + \symup{\Delta}n \frac{T \cdot p_0}{T_0 \cdot (p - p^\prime)}
  \label{eq:3}
\end{equation}
\section{Durchführung}
\subsection{Versuchsaufbau}
\section{Auswertung}
\subsection{Wellenlänge eines Helium-Neon-Lasers}
\begin{table}[h]
  \centering
  \caption{Messwerte für die Messung an einem Helium-Neon-Laser sowie für jedes Wertepaar
  berechnete Wellenlänge.}
  \begin{tabular}{c c c}
    \toprule
    $\symup{\Delta}d$ / \si{\milli\metre} & $z$ & $\lambda$ \\
    \midrule
    2 & 1286 & 619.98 \\
    2 & 1327 & 600.82 \\
    2 & 1284 & 620.94 \\
    2 & 1305 & 610.95 \\
    2 & 1340 & 594.99 \\
    2 & 1288 & 619.01 \\
    2 & 1326 & 601.27 \\
    2 & 1300 & 613.30 \\
    2 & 1330 & 599.47 \\
    2 & 1292 & 617.10 \\
    \bottomrule
  \end{tabular}
  \label{tab:1}
\end{table}

Die Messwerte finden sich in Tabelle \ref{tab:1}. Ebenfalls in dieser Tabelle finden
sich die nach \eqref{eq:1} berechneten Wellenlängen. Die an der Millimeterschraube
gemessenen Verschiebungen müssen durch die Hebelübersetzung von \num{5.017} geteilt
werden, um die tatsächliche Verschiebung des Spiegels zu erhalten.
Durch Mittelwertbildung erhält man einen Wert von:
\begin{align*}
  \lambda = \SI{609.8(31)}{\nano\metre}.
\end{align*}

\subsection{Brechungsindex von Luft und CO2}

\begin{table}[h]
  \centering
  \caption{In Tabelle \subref{tab:2} sind die gemessenen Werte bei Füllung der
  Kammer mit Luft, in Tabelle \subref{tab:3} für CO2 eingetragen. Außerdem
  eingetragen ist der für jedes Wertepaar berechnete Brechungsindex $n$.}
  \label{tab:4}
    \begin{subtable}{0.49\textwidth}
    \centering
    \begin{tabular}{c c c}
      \toprule
      $P$ / \si{\bar} & $z$ & $n$ \\
      \midrule
      0.2132 & 46 & 1.000416 \\
      0.2132 & 33 & 1.000298 \\
      0.2132 & 47 & 1.000425 \\
      0.2132 & 32 & 1.000289 \\
      0.2132 & 37 & 1.000334 \\
      0.2132 & 35 & 1.000316 \\
      0.2132 & 33 & 1.000298 \\
      0.2132 & 50 & 1.000452 \\
      0.2132 & 34 & 1.000307 \\
      0.2132 & 33 & 1.000298 \\
      \bottomrule
    \end{tabular}
    \caption{Werte für Luft.}
    \label{tab:2}
  \end{subtable}
    \begin{subtable}{0.49\textwidth}
    \centering
    \begin{tabular}{c c c}
      \toprule
      $P$ / \si{\bar} & $z$ & $n$ \\
      \midrule
      0.2132 & 52 & 1.000470 \\
      0.2132 & 50 & 1.000452 \\
      0.2132 & 50 & 1.000452 \\
      0.2132 & 51 & 1.000461 \\
      0.2132 & 53 & 1.000479 \\
      0.2132 & 52 & 1.000470 \\
      0.2132 & 50 & 1.000452 \\
      0.2132 & 50 & 1.000452 \\
      0.2132 & 52 & 1.000470 \\
      0.2132 & 52 & 1.000470 \\
      \bottomrule
    \end{tabular}
    \caption{Werte für CO2.}
    \label{tab:3}
    \end{subtable}
\end{table}

Zur Bestimmung der Brechnungsindize der beiden Gase nach \eqref{eq:3} (mit \eqref{eq:2})
sind einige Werte notwendig. Diese sind wie folgt:
\begin{align*}
  \text{Normaldruck: } p_0 =& \; \SI{1.0132}{\bar} \\
  \text{Normaltemperatur: } T_0 =& \; \SI{273.15}{\kelvin} \\
  \text{Umgebungstemperatur: } T =& \; \SI{293.15}{\kelvin} \\
  \text{Durchmesser der Zelle: } b =& \; \SI{50}{\milli\metre} \\
  \text{Wellenlänge Laser: } \lambda =& \; \SI{635}{\nano\metre}. \\
\end{align*}
Weiter ist zu beachten, dass sich die Kammer in allen Fällen nur auf \SI{-0.8}{\bar}
evakuieren ließ, sodass der verbeleibende Kammerinnendruck bei \SI{0.2132}{\bar}
lag. Der Nullpunkt der Skala des Manometers ist auf $p_0$ geeicht. Die gemessenen
Werte sowie die nach \eqref{eq:3} bestimmten Brechungsindize sind in Tabelle \ref{tab:4}
dargestellt. Der in den Rechnungen benötigte Wert $\symup{\Delta}n$ wird nach \eqref{eq:2}
bestimmt. Aufgrund der großen Abweichung zwischen dem auf dem Laser angegebenen und dem
oben bestimmten Wert für $\lambda$ wird ersterer in den Rechnungen genutzt. Es ergeben
sich folgende Mittelwerte:
\begin{align*}
  n_\symup{Luft} =& \; \num{1.000343(20)} \\
  n_\symup{CO2} =& \; \num{1.000463(3)}. \\
\end{align*}

\section{Diskussion}
Im Vergleich mit dem auf dem Laser angebenen Wert von \SI{635}{\nano\metre} für die
Wellenlänge zeigen sich Abweichungen, die nicht im Bereich der Messungenauigkeiten liegen.
Es ist daher von einem systematischen Fehler auszugehen. Denkbar sind vorallem
Fehlzählungen. Auch wenn die Geschwindigkeit, mit der der Spiegel verstellt wurde,
extrem gering war, kann nicht ausgeschlossen werden, dass die Frequenz, mit der
die resultierenden Maxima am Photodetektor vorbeizogen das Auflösungsvermögen eben
dieses überstiegen.\\
Bei Vergleich zwischen den für die Brechungsindize gemessenen Werten mit Literaturwerten
von \num{1.0002765}\cite{luft} für Luft und \num{1.0004476}\cite{co2} für CO2 zeigen
sich ebenfalls keine Übereinstimmungen im Fehlerintervall. Die relativen Abweichungen
sind mit \num{4.65e-3} und \SI{1.24e-3}{\percent} jedoch extrem gering. Auch hier sind
Fehlzählungen das Wahrscheinlichste. Eine erneute Messung mit einem genaueren Messgerät
sollte daher alle Werte verbessern können.
\newpage
\nocite{*}
\printbibliography
