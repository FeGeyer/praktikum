\maketitle
\setcounter{page}{1}
\tableofcontents
\newpage
\pagenumbering{arabic}
\section{Theorie}
Ziel des Versuchs ist die expermentelle Betrachtung des Röntgenemissionsspektrums
von Kupfer sowie verschiedener Absorbtionsspektren.\\
\noindent
Röntgenstrahlung entsteht bei der Wechselwirkung von beschleunigten Elektronen
mit Materie. Die Elektronen werden dabei in einer evakuiertien Röhre (der sogenannten
Röntgenröhre) unter Zuhilfenahme des glühelektrischen Effekts aus einer Kathode ausgelöst
und zu einer Anode hin beschleunigt. Beim Auftreffen auf das Anodenmaterial führen
zwei Prozesse zur Entstehung von Röntgenstrahlung:
\begin{enumerate}
  \item [1.: Bremsstrahlung] Die Elektronen treten hierbei in Wechselwirkung mit den
  Coulombfeldern der Atomkerne des Anodenmaterials. Die durch den dabei folgenden
  Abbremsvorgang verloren gegenagene Energie wird in Form eines Photons emittiert.
  Das resultierende Spektrum ist kontinuierlich. Es weist eine Grenzwellenlänge auf,
  unter der keine Röntgenstrahlung gemessen werden kann. Sie berechnet sich zu
  \begin{equation}
    \lambda_{\symup{min}} = \frac{hc}{e_0U}
  \end{equation}
  (mit Plankschem Wirkungsquantum $h$, Vakuumlichtgeschwindigkeit $c$ und Elektronenruhemasse $e_0$)
  und entspricht der vollständigen Abbremsung des Elektrons, bei dem die
  aus der Beschleunigung gewonnene gesamte kinetische Energie umgewandelt wird.
  Dieses Spektrum ist kontinuierlich, es wird daher oft als kontinuierliches Bremsspektrum
  oder auch als Bremsberg bezeichnet.
  \item [2. Charakteristisches Spektrum]
\end{enumerate}
\section{Durchführung}
\subsection{Versuchsaufbau}
\subsection{Versuchsdurchführung}
\section{Auswertung}
\section{Diskussion}
\newpage
\nocite{*}
\printbibliography
