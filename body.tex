\maketitle
\tableofcontents
\newpage

\section{Zielsetzung}
In diesem Versuch geht es um Messung der Schwingungs- und Schwebungsdauer von gekoppelten Pendeln.
Untersucht werden gleich- und gegensinnige sowie gekoppelte Schwingungen.
\section{Theorie}
Ein einzelndes Fadenpendel mit einem Faden der Länge $\textit{l}$, Masse $\textit{m}$, welches reibungsfrei aufgehängt wurde,
schwingt für kleine Auslenkungen (sin $\phi \approx \phi$) mit der Schwingungsfrequenz:
\begin{equation}
  \omega = \sqrt{\frac{\textit{l}}{g}}
  \label{e1}
\end{equation}
Mit \eqref{e1} und
\begin{equation*}
  \textit{T} = \frac{2\pi}{\omega}
\end{equation*}
ergibt sich als Formel für die gesuchte Schwingungsdauer:
\begin{equation}
  \textit{T} = 2\pi \sqrt{\frac{g}{\textit{l}}}
\end{equation}
\section{Aufbau und Durchführung}
  \subsection{Aufbau}
  \subsection{Durchführung}
\newpage

\section{Auswertung}

\newpage
\nocite{*}
\printbibliography
