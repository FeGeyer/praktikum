\maketitle
\setcounter{page}{1}
\tableofcontents
\newpage
\pagenumbering{arabic}
\section{Theorie}
Ziel des Versuchs ist die Betrachtung der $\textsc{Frauenhofer}$-Lichtbeugung an Parallel- und
Doppelspalten.
Optische Beugung tritt immer dann auf, wenn Licht auf ein Objekt trifft, dessen
Abmessungen kleiner sind, als der Durchmesser des Lichtstrahls selbst. Zur Erklärung
von Beugungsphänomenen wird das $\textsc{Huygens}$sche Prinzip der Elementarwellen genutzt.
Desweiteren muss unterschieden werden, ob die Lichtquelle im endlichen liegt und der
Lichtstrahl auseinander läuft, oder ob die Lichtquelle so weit von der Beugungsebene
entfernt liegt, dass der Lichtstrahl als paralleles Bündel auftrifft. Es wird dann
davon gesprochen, dass die Lichtquelle "ins Unendliche verschoben wurde". Der erste Fall wird
als $\textsc{Fresnel}$-Beugung bezeichnet, der Zweite als $\textsc{Frauenhofer}$-Beugung
(siehe Abbildung \ref{abb:1}).\\
Die $\textsc{Frauenhofer}$-Beugung wird hier am Parallelspalt diskutiert. Es gelten
folgende Annahmen.
\begin{itemize}
  \item Die Spaltbreite ist gering gegenüber der Spaltlänge, sodass der Strahl nur in
  seiner Breite begrenzt wird.
  \item Es werden nur Strahlen betrachtet, die unter dem selben Winkel gebeugt wurden.
  Um dies zu realisieren, muss der Abstand zwischen Beobachtungsebene groß im Vergleich
  mit der Spaltbreite sein. Dafür wird häufig eine Sammellinse in den Strahlengang nach
  der Beugungebene gebracht.
  \item Es treffen ebene Wellen der Form
  \begin{equation}
    A(z,t) = A_0 \exp \left ( \symup{i} \left(\omega t - \frac{2 \pi z}{\lambda}\right) \right)
    \label{eq:1}
  \end{equation}
  mit Wellenlänge $\lambda$ auf die Beugungsebene auf. Um dies zu realisieren ist
  kohärentes Licht notwendig, dass "aus dem Unendlichen" kommend auf den Spalt trifft.
  Als Quelle solchen Lichtes werden beispielsweise Laser genutzt.
\end{itemize}
Alle Punkte der eintreffenden Wellenfront sind nun nach dem $\textsc{Huygens}$schen
Prinzip Ausgangspunkt einer sich kugelförmig ausbreitenden Elementarwelle. Trifft der
Strahl auf den Spalt führt dies dazu, dass der Lichtstrahlen in Teilstrahlen
aufgeteilt wird, die untereinander interferieren, da es einen Gangunterschied $s$ zwischen ihnen
gibt (siehe Abbildung \ref{abb:2}) Aus diesem Gangunterschied lässt sich eine Phasenunterschied
$\delta$ betimmen:
\begin{equation}
  \delta = \frac{2 \pi s}{\lambda} = \frac{2 \pi x \sin(\varphi)}{\lambda}.
  \label{eq:2}
\end{equation}
Aus \eqref{eq:1} und \eqref{eq:2} lässt sich nun ein Ausdruck für die Amplitude $B$
in Abhängigkeit des Winkels finden:
\begin{equation}
  B(\varphi) = A_0 b \, \symup{sinc}(\eta)
\end{equation}
mit
\begin{equation}
  \eta = \frac{\pi b \sin(\varphi)}{\lambda}.
\end{equation}
Die Amplitude ist jedoch nicht messbar, da die Frequenz des Lichtes extrem hoch
und daher für Messgeräte nicht auflösbar ist. Messbar ist jedoch die Intensität,
die dem Quadrat der Amplitudenfunktion entspricht:
\begin{equation}
  I(\varphi) \propto B(\varphi)^2 = A_0^2 b^2 \, \symup{sinc}^2(\eta).
  \label{eq:3}
\end{equation}
Diese Funktion wird auch als Beugungsfigur bezeichnet.\\
Wie für den Einzelspalt lässt sich eine Beugungsfigur auch für den Doppelspalt definieren.
Hier muss jedoch der Gangunterschied $s$ anders definiert werden (siehe Abbildung \ref{abb:3}).
Man erhält:
\begin{equation}
  B(\varphi)^2 = A_0^2 \cos^2\left(\frac{\pi s \sin(\varphi)}{\lambda} \right) \, \symup{sinc}^2(\eta).
  \label{eq:4}
\end{equation}
\section{Durchführung}
\subsection{Versuchsaufbau}
\subsection{Versuchsdurchführung}
\section{Auswertung}
\section{Diskussion}
\newpage
\nocite{*}
\printbibliography
