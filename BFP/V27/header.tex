\documentclass[titlepage=firstiscover, bibliography=totoc, captions=tableheading]{scrartcl}
\titlehead{
  \centering
  \includegraphics[scale=2.3]{logo.png}
}
\author{Felix Geyer \\
  \texorpdfstring{\href{mailto:felix.geyer@tu-dortmund.de}{felix.geyer@tu-dortmund.de}\and}{,}
  Rune Dominik \\
  \texorpdfstring{\href{mailto:rune.dominik@tu-dortmund.de}{rune.dominik@tu-dortmund.de}}{}
  }
\title{V27: Der Zeeman-Effekt}
\publishers{TU Dortmund - Fakultät Physik}
\date{Durchführung: 27. November 2017 \\
      Abgabe: xx. xxxx 2017}
\usepackage[aux]{rerunfilecheck}
\usepackage{polyglossia}
\setmainlanguage{german}
\usepackage{amsmath}
\usepackage{amssymb}
\usepackage{mathtools}
\usepackage{fontspec}
\usepackage[version=4]{mhchem}
\usepackage{scrhack}
\usepackage{float}
\floatplacement{table}{htbp}
\floatplacement{figure}{htbp}

\usepackage[locale=DE, separate-uncertainty=true, per-mode=symbol-or-fraction, decimalsymbol=.]{siunitx}
%\usepackage{siunitx}

\usepackage[style=alphabetic]{biblatex}
\addbibresource{lit.bib}

\usepackage[section, below]{placeins}
\usepackage[labelfont=bf,
font=small,
width=0.9\textwidth,
format=plain,
indention=1em]{caption}
\usepackage{graphicx}
\usepackage{grffile}
\usepackage{subcaption}

\usepackage[math-style=ISO, bold-style=ISO, sans-style=italic, nabla=upright, partial=upright]{unicode-math}
\setmathfont{Latin Modern Math}

\usepackage[autostyle]{csquotes}

\usepackage[unicode]{hyperref}

\usepackage{bookmark}

\usepackage{booktabs}

\usepackage{ulem}
