\section{Auswertung}
Alle im folgenden durchgeführten Ausgleichsrechnungen werden mit der Funktion\\
\texttt{curve\_fit} aus der Python\cite{python}-Bibliothek
\texttt{scipy.optimize}\cite{scipy} ausgeführt.
Fehlerrechnungen werden durch die Bibliothek \texttt{uncertainties}\cite{uncertainties} automatisiert,
Mittelwerte werden durch die Funktion \texttt{numpy.mean}\cite{numpy}, Fehler des Mittelwertes
durch \texttt{sem} aus \texttt{scipy.stats}.
Grafiken werden durch die Bibliothek \texttt{matplotlib}\cite{matplotlib}.

\subsection{Überprüfung der Stabilitätsbedingung}
Es werden zwei Resonatorkonfigurationen überprüft, diese haben die folgenden
Spiegelanordnungen:
\begin{align*}
  1: &\; 1400\,\text{mm}\,/\,\text{flat} \; \text{HR} \; &+& \; 1400\,\text{mm}\,/\,\text{flat} \; \text{OC}\\
  2: &\; \text{flat}\,/\,\text{flat} \; \text{HR} \; &+& \; 1400\,\text{mm}\,/\,\text{flat} \; \text{OC}.
\end{align*}

Theoretisch ergeben sich für $g_1 g_2$ mit $g_i = 1 - \frac{L}{r_i}$ die in
\autoref{abb:Aus1} dargestellten Kurven.
Für beide Spiegelkonfigurationen ist für eine Resonatorlänge von \SI{1400}{\milli\metre}
die Stabilitätsbedingung \eqref{eqn:Stabilität} verletzt.
Im Experiment zeigt sich ein Aussetzen der Lasingtätigkeit für eine Resonatorlänge von
\begin{align*}
  1: &\; \SI{1382}{\milli\metre}\\
  2: &\; \SI{1387}{\milli\metre}.
\end{align*}

\begin{figure}[h]
  \centering
  \includegraphics[width=\textwidth]{Auswertung/Plots/g1g2.pdf}
  \caption{Abhängigkeit des Produktes $g_1 g_2$ von der Resonatorlänge mit eingetragenen
  Grenzen der Stabilitätsbedingung.}
  \label{abb:Aus1}
\end{figure}

\subsection{Vermessung der TEM Moden}
Die Intensitätsverteilung der TEM Moden $T_{\symup{00}}$ und $T_{\symup{01}}$
folgt den Verteilungen
\begin{align*}
  I_{\symup{00}}(x) &= I_{0} \exp{\left(-\frac{(x-x_{0})^{2}}{\sigma^{2}}\right)} \text{ und}\\
  I_{\symup{01}}(x) &= I_{0} (x-x_{0})^{2} \exp{\left(-\frac{(x-x_{0})^{2}}{\sigma^{2}}\right)},
  \label{eq:Aus2}
\end{align*}
wobei $x_{0}$ eine kontinuierliche Verschiebung der Funktionen auf der $x$-Achse,
$I_{0}$ die Maximalintensität und $\sigma$ die Standartabweichung der Verteilungen
darstellt.
Die nach den oben angegebenen Funktionen durchgeführten Ausgleichsrechnungen durch
die in \autoref{tab:Aus2} und \autoref{tab:Aus3} gegebenen Daten finden sich in
\autoref{abb:Aus3} und \autoref{abb:Aus4}.
Für die 00-Mode ergeben sich die Parameter
\begin{align*}
  I_{0} &= \SI{205(4)}{\nA}\\
  x_{0} &= \SI{10.76(21)}{\mm}\\
  \sigma &= \SI{17.0(5)}{\per\mm\squared}
\end{align*}
und für die 01-Mode die Parameter
\begin{align*}
  I_{0} &= \SI{0.626(24)}{\nA}\\
  x_{0} &= \SI{14.16(11)}{\mm}\\
  \sigma &= \SI{-10.75(17)}{\per\mm\squared}.
\end{align*}

\begin{table}[h]
  \centering
  \caption{Messwerte für die $T_{\symup{00}}$-Mode.}
  \label{tab:Aus2}
  \begin{tabular}{c c | c c}
    \toprule
    $x$ / \si{\mm} &
    $I$ / \si{\uA} &
    $x$ / \si{\mm} &
    $I$ / \si{\uA}\\
    \midrule
    0.0 & 90.0 & 15.0 & 190.0 \\
0.5 & 110.0 & 15.5 & 185.0 \\
1.0 & 100.0 & 16.0 & 185.0 \\
1.5 & 108.0 & 16.5 & 180.0 \\
2.0 & 125.0 & 17.0 & 170.0 \\
2.5 & 130.0 & 17.5 & 165.0 \\
3.0 & 135.0 & 18.0 & 155.0 \\
3.5 & 160.0 & 18.5 & 145.0 \\
4.0 & 150.0 & 19.0 & 135.0 \\
4.5 & 160.0 & 19.5 & 130.0 \\
5.0 & 180.0 & 20.0 & 120.0 \\
5.5 & 190.0 & 20.5 & 110.0 \\
6.0 & 190.0 & 21.0 & 100.0 \\
6.5 & 195.0 & 21.5 & 90.0 \\
7.0 & 200.0 & 22.0 & 82.0 \\
7.5 & 190.0 & 22.5 & 75.0 \\
8.0 & 205.0 & 23.0 & 70.0 \\
8.5 & 210.0 & 23.5 & 60.0 \\
9.0 & 150.0 & 24.0 & 53.0 \\
9.5 & 150.0 & 24.5 & 45.0 \\
10.0 & 130.0 & 25.0 & 43.0 \\
10.5 & 200.0 & 25.5 & 37.0 \\
11.0 & 195.0 & 26.0 & 33.0 \\
11.5 & 205.0 & 26.5 & 27.0 \\
12.0 & 205.0 & 27.0 & 23.0 \\
12.5 & 205.0 & 27.5 & 20.0 \\
13.0 & 200.0 & 28.0 & 18.5 \\
13.5 & 200.0 & 28.5 & 15.3 \\
14.0 & 205.0 & 29.0 & 14.0 \\
14.5 & 200.0 & 29.5 & 12.0 \\

    \bottomrule
  \end{tabular}
\end{table}

\begin{table}[h]
  \centering
  \caption{Messwerte für die $T_{\symup{01}}$-Mode.}
  \label{tab:Aus3}
  \begin{tabular}{c c | c c}
    \toprule
    $x$ / \si{\mm} &
    $I$ / \si{\uA} &
    $x$ / \si{\mm} &
    $I$ / \si{\uA}\\
    \midrule
    6.5 & 10.0 & 18.5 & 9.5 \\
7.0 & 10.0 & 19.0 & 10.0 \\
7.5 & 15.0 & 19.5 & 12.0 \\
8.0 & 15.0 & 20.0 & 13.0 \\
8.5 & 15.0 & 20.5 & 14.0 \\
9.0 & 12.0 & 21.0 & 14.0 \\
9.5 & 10.0 & 21.5 & 14.5 \\
10.0 & 9.0 & 22.0 & 15.0 \\
10.5 & 7.0 & 22.5 & 15.0 \\
11.0 & 6.0 & 23.0 & 14.0 \\
11.5 & 4.0 & 23.5 & 13.5 \\
12.0 & 3.0 & 24.0 & 13.0 \\
12.5 & 2.0 & 24.5 & 12.5 \\
13.0 & 1.5 & 25.0 & 12.0 \\
13.5 & 1.2 & 25.5 & 10.5 \\
14.0 & 1.2 & 26.0 & 10.0 \\
14.5 & 1.3 & 26.5 & 6.0 \\
15.0 & 1.2 & 27.0 & 5.5 \\
15.5 & 2.5 & 27.5 & 5.0 \\
16.0 & 3.0 & 28.0 & 5.0 \\
16.5 & 4.5 & 28.5 & 4.5 \\
17.0 & 5.0 & 29.0 & 4.0 \\
17.5 & 7.0 & 29.5 & 4.0 \\
18.0 & 8.5 & 30.0 & 3.5 \\

    \bottomrule
  \end{tabular}
\end{table}

\begin{figure}[h]
  \centering
  \includegraphics[width=\textwidth]{Auswertung/Plots/M00.pdf}
  \caption{Vermessung der 00-Mode mit Ausgleichsrechnung.}
  \label{abb:Aus3}
\end{figure}

\begin{figure}[h]
  \centering
  \includegraphics[width=\textwidth]{Auswertung/Plots/M01.pdf}
  \caption{Vermessung der 01-Mode mit Ausgleichsrechnung.}
  \label{abb:Aus4}
\end{figure}


\subsection{Untersuchung der Polarisation}
Nach dem Gesetz von Malus verhält sich die Intensität linear polarisierten Lichts
mit Eingangsintensität $I_{0}$ nach Durchgang durch den Polarisator in Stellung $\varphi$ wie
\begin{equation}
  I(\varphi) = I_{0} \cos{\varphi}^{2}.
  \label{eq:Aus1}
\end{equation}
Aufgrund der Verwendung von Brewsterfenstern ist von einer anfänglichen Polarisation
des Laserstrahls auszugehen.
Der am Brewsterfenster reflektierte Strahl ist rein s-polarisiert, während der
transmittierte Teil sowohl s- als auch p-Polarisation aufweist. Mit jedem Durchgang
durch das Fenster erleidet der s-polarisierte Teil große Verluste, wodurch schnell
die zweite Stabilitätsbedingung an den Laser verletzt wird, nähmlich dass der
Energieverlust den Energiegewinn nicht übersteigen darf.
Das den Laser verlassende Licht ist somit quasi perfekt linear polarisiert.

Werden die in \autoref{tab:Aus1} gezeigten Daten mit einer modifizierten Version
von \eqref{eq:Aus1}
\begin{equation*}
  I(\varphi) = I_{0} \cos{\varphi + \varphi_{0}}^{2},
\end{equation*}
welche die unbekannte, interne Verschiebung $\varphi_{0}$ des Polarisators berücksichtigt,
gefittet, ergibt sich der in \autoref{abb:Aus2} gezeigte Verlauf.
Die freien Parameter der Ausgleichsrechnung bestimmen sich zu
\begin{align*}
  I_{0} &= \SI{25.4(16)}{\uA} \\
  \varphi_{0} &= \SI{72.2(31)}{\degree}.
\end{align*}

\begin{figure}[h]
  \centering
  \includegraphics[width=\textwidth]{Auswertung/Plots/Winkel.pdf}
  \caption{Polarisationsmessung mit Ausgleichsfunktion.}
  \label{abb:Aus2}
\end{figure}

\begin{table}[h]
  \centering
  \caption{Messwerte der Polarisationsbestimmung.}
  \label{tab:Aus1}
  \begin{tabular}{c c}
    \toprule
    $\varphi$ / ° &
    $I$ / \si{\uA}\\
    \midrule
    0 & 6.00 \\
20 & 13.50 \\
40 & 23.50 \\
60 & 23.00 \\
80 & 20.00 \\
100 & 15.00 \\
120 & 7.50 \\
140 & 1.15 \\
160 & 0.40 \\
180 & 3.00 \\
200 & 7.50 \\
220 & 12.00 \\
240 & 25.00 \\
260 & 28.00 \\
280 & 25.00 \\
300 & 18.50 \\
320 & 7.00 \\
340 & 0.90 \\

    \bottomrule
  \end{tabular}
\end{table}

\subsection{Wellenlängenbestimmung}
Zur Wellenlängenbestimmung wird der Laserstrahl an einem Gitter mit Gitterkonstante
$g = \SI{0.01}{\per\mm}$ gebeugt, das im Abstand $L=\SI{13.6}{\cm}$ vor einer
Photodiode steht.
Die Messwerte finden sich in \autoref{tab:Aus4}, diese sind mit den durch
die \texttt{scipy}-Funktion \texttt{find\_peaks} gefundenen Maxima in
\autoref{abb:Aus5} abgebildet.
Für die Maxima und die resultierenden Wellenlängen
folgen die in \autoref{tab:Aus5} zu findenden Werte, die Abstände zum Maximum wurden
dabei über
\begin{equation}
  \alpha = \tan{\left(\frac{\symup{\Delta}x}{L}\right)}^{-1}
\end{equation}
in einen Winkel und da es sich um Maxima erster Ordnung handelt über
\begin{equation}
  \lambda = d \sin{(\alpha)}
\end{equation}
in eine Wellenlänge umgerechnet.
Im Mittel ergibt sich eine Wellenlänge von \\
\SI{642.05(1827)}{\nm}.

\begin{table}[h]
  \centering
  \caption{Werte der Maxima.}
  \label{tab:Aus5}
  \begin{tabular}{c c c c}
    \toprule
    $x$ / \si{\mm} &
    $\symup{\Delta}x$ / \si{\mm} &
    $\alpha$ / \si{\degree} &
    $\lambda$ / \si{\nm}\\
    \midrule
    -0.5 & 9.0 & 0.0661 & 660.32 \\
    0 & 0 & 0 & - \\
    -18 & 8.5 & 0.0624 & 623.78 \\
    \bottomrule
  \end{tabular}
\end{table}

\begin{table}[h]
  \centering
  \caption{Messwerte für die Wellenlängenbestimmung.}
  \label{tab:Aus4}
  \begin{tabular}{c c | c c}
    \toprule
    $x$ / \si{\mm} &
    $I$ / \si{\uA} &
    $x$ / \si{\mm} &
    $I$ / \si{\uA}\\
    \midrule
    1.5 & 0.035 & -9.5 & 84.000 \\
1.0 & 0.120 & -10.0 & 30.000 \\
0.5 & 0.850 & -10.5 & 2.000 \\
0.0 & 2.500 & -11.0 & 0.350 \\
-0.5 & 7.600 & -11.5 & 0.175 \\
-1.0 & 6.800 & -12.0 & 0.115 \\
-1.5 & 2.800 & -12.5 & 0.090 \\
-2.0 & 0.780 & -13.0 & 0.065 \\
-2.5 & 0.160 & -13.5 & 0.050 \\
-3.0 & 0.130 & -14.0 & 0.040 \\
-3.5 & 0.110 & -14.5 & 0.033 \\
-4.0 & 0.110 & -15.0 & 0.033 \\
-4.5 & 0.115 & -15.5 & 0.026 \\
-5.0 & 0.130 & -16.0 & 0.030 \\
-5.5 & 0.140 & -16.5 & 0.050 \\
-6.0 & 0.140 & -17.0 & 0.200 \\
-6.5 & 0.150 & -17.5 & 0.500 \\
-7.0 & 0.270 & -18.0 & 0.900 \\
-7.5 & 0.400 & -18.5 & 0.330 \\
-8.0 & 1.800 & -19.0 & 0.090 \\
-8.5 & 10.000 & -19.5 & 0.011 \\
-9.0 & 30.000 & -20.0 & 0.017 \\

    \bottomrule
  \end{tabular}
\end{table}

\begin{figure}[h]
  \centering
  \includegraphics[width=\textwidth]{Auswertung/Plots/Beugung.pdf}
  \caption{Beugungsbild mit eingezeichneten Maxima zur Wellenlängenbestimmung.}
  \label{abb:Aus5}
\end{figure}

\subsection{Vermesung der Longitudinalen Moden}
Die gemessenen Peaks des Spektrums finden sich in \autoref{tab:Aus5}, zusammen
mit den gemittelten Peakabständen.
Bei einer Temperatur $T = \SI{300}{\K}$ folgt nach der Boltzman-Verteilung
mit Boltzman-Konstante $k$ und Neon-Atomgewicht$M = \SI{20.18}{\atomicmassunit}$\cite{Neon}
eine mittlere Teilchengeschwindigkeit von
\begin{equation*}
  < v > = \sqrt{\frac{2 k T}{M}} = \SI{497.20}{\km\per\s}
\end{equation*}
und damit bei Vakuumlichtgeschwindigkeit $c$ eine Dopplerverschiebung von
\begin{equation*}
  2  (f_{\text{Ruhe}} - \frac{c + <v>}{c - <v>}  f_{\text{Ruhe}}) = \SI{3.14}{\GHz}.
\end{equation*}

\begin{table}[h]
  \centering
  \caption{Peakpositionen für 3 Resonatorlängen, die Positionen sind in \si{MHz}
  angegeben.}
  \label{tab:Aus5}
  \begin{tabular}{c | c c c c c c c c c c c | c}
    \toprule
    $L$ / \si{\cm} & & & & & & & & & & & & $\symup{\Delta}\nu$ / \si{MHz}\\
    \midrule
    \num{68.35} & 229 & 462 & 691 & 924 & 1153 & 1382 & -   & -   & -    & -    & -    & \num{230.6(10)}\\
    \num{129.5} & 118 & 233 & 351 & 470 & 584  & 703  & 817 & 936 & 1054 & 1169 & 1287 & \num{116.9(7)}\\
    \num{137.5} & 110 & 221 & 331 & 442 & 553  & 663  & 774 & 884 & 995  & 1102 & 1212 & \num{110.2(4)}\\
    \bottomrule
  \end{tabular}
\end{table}

\section{Diskussion}
\subsection{Überprüfung der Resonatorlänge}
Für beide Konfigurationen konnte der theoretische Wert annähernd erreicht bleiben.
Aufgrund der Empfindlichkeit des Versuchsaufbaus gegenüber kleinster Verrückungen
senkrecht zur Bewegungsrichtung bei großen Resonatorlängen sollten diese
Ergebnisse jedoch mit einem stabileren Aufbau erneut überprüft werden.
Da die theoretischen Werte bei Durchführung des Versuchs bekannt waren und die
Lasingtätigkeit des Lasers während der Verrückung aufgrund leichter Wackelbewegungen
immer wieder aussetzt, verbleibt die Möglichkeit, dass ein Aussetzen des Lasens
durch Wackeln fälschlicher Weise als Messwert gewertet wurde.
Insbesondere die erste untersuchte Konfiguration berührt im optimalen Falle
am untersuchten Punkt lediglich den $0$-Wert berührt und eventuelle Unterschreitungen
von kleinen Justageabweichungen herrühren.

\subsection{Vermessung der TEM Moden}
Beide Kurven lassen sich mit kleinen Fehlern auf die Theoriekurve anpassen,
sodass die Vermessung der Moden allgemein als erfolgreich angenommen werden kann.
Lediglich die Vermessung der 01-Mode gestalltete sich dahingehend als schwierig,
dass durch einen schiefen Draht, der als Modenblende genutzt wurde, eine Schiefstellung
des aufgenommenen Bildes resultierte. Da diese durch die zur Aufnahme genutzten
Photodiode nicht kompensiert werden konnte, konnten bei kleinen Werten auf der
$x$-Achse keine brauchbaren Messdaten aufgenommen werden.
Für eine statistisch besser unterlegte Ausgleichsrechnung ist die Messung mit
einer besseren Modenblende zu wiederholen.

\subsection{Untersuchung der Polarisation}

\subsection{Wellenlängenbestimmung}

\subsection{Vermessung der Longitudinalen Moden}
