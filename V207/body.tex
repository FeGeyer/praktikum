\maketitle
\tableofcontents
\newpage

\section{Zielsetzung}
Ziel des Versuches ist die Bestimmung der Temperaturabhängigkeit der dynamischen
Viskosität von destilliertem Wasser mithilfe des Kugelfall-Viskosimeters nach Höppler.
\section{Theorie}
Bei Bewegung durch Flüssigkeiten erfahren Körper eine Reibungskraft, die von der
Berührungsfläche A und der Geschwindigkeit $v$ abhängt. Eine weitere wichtige Rolle
spielt die $\textbf{dynamische Viskosität}$, eine Eigenschaft der Flüssigkeit, die
temperaturabhängig ist. Mithilfe des $\textbf{Kugelfall-Viskosimeters nach Höppler}$
lässt sich Diese bestimmen. Wenn der Radius der Kugel sich nur marginal von dem
der Röhre unterscheidet, sodass sich keine Turbulenzen bilden, dann gilt für die
Stokessche Reibung
\begin{equation}

\end{equation}
\section{Durchführung}

\subsection{Versuchsaufbau}

\subsection{Versuchsdurchführung}

\section{Fehlerrechnung}
Es gibt:
\begin{equation}
  \bar{T} = \frac{1}{n} \sum_{i=1}^{n} T_{i}
  \label{eqn:1}
\end{equation}
den Mittelwert und:
\begin{equation}
  \sigma_{\bar{T}} = \sqrt{\frac{1}{n(n-1)} \sum_{i=1}^{n}(\bar{T}-T_i)^2}
  \label{eqn:2}
\end{equation}
den Fehler des Mittelwertes. Falls zwei fehlerbehaftete Größen in einer Gleichung
zur Bestimmung einer anderen Größe Verwendung finden, dann berechnte sich der Gesamtfehler
nach der Gaußschen Fehlerfortpflanzung zu
\begin{equation}
    \symup \Delta f(x_1, x_2, ..., x_n) = \sqrt{\left(\frac{\symup df}{\symup dx_1} \symup \Delta
    x_1 \right)^2 +    \left(\frac{\symup df}{\symup dx_2} \symup \Delta
    x_2 \right)^2 + ... + \left(\frac{\symup df}{\symup dx_n} \symup \Delta x_n \right)^2} \ .
    \label{eqn:3}
\end{equation}

\section{Auswertung}

\section{Diskussion}
\newpage
\nocite{*}
\printbibliography
