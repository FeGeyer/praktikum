\section{Auswertung}
\subsection{Messung des Erdmagnetfeldes}
Wie in der Durchführung bereits beschrieben, musste die horizontale Komponente
des Erdmagnetfeldes kompensiert werden. Während der Messung wurde die Stärke des
gesamten Horizontalfeldes (Sweepfeld + Horizontal) in Abhängigkeit von der Resonanzfrequenz von beiden
Isotope gemessen. Zu diesem Zweck wurde der Strom erhöht, bis die jeweiligen
Resonanzstellen zu sehen waren. Mit der Helmholtz-Gleichung
\begin{equation}
  B = \symup{\mu_0} \, \frac{8 \, I \, N}{\sqrt{125}\, R} \, ,
  \label{eqn:helmholtz}
\end{equation}
wobei $I$ der Strom, $N$ die Windungszahl der Helmholtzspule und $R$ der Spulenradius
ist, werden die Magnetfelder des Sweep-Feldes und des Horizontalfeldes ausgerechnet
und addiert. Das Ergebnis wird gegen die RF-Frequenz aufgetragen, wie in \autoref{abb:Bfelf}
zu sehen.
\begin{figure}
  \centering
  \includegraphics[scale=0.7]{Auswertung/Plots/Bfeld.pdf}
  \caption{Stärke des Horizontalfeldes gegen die Resonanzfrequenz aufgetragen.
  Zusätzlich ist ein linearer Fit geplottet. Die erste Resonanzstelle wurde Isotop 1
  zugeordnet, die zweite Resonanzstelle Isotop 2.}
  \label{abb:Bfelf}
\end{figure}
Aus den y-Achsenabschnitten der Fits lässt sich die Stärke des horizontalen
Erdmagnetfeldes ablesen. Die Parameter des linearen Fits sind in \autoref{tab:Fit-BFeld}
zu sehen.
\begin{table}
  \centering
  \caption{Parameter des linearen Fits $f(x) = mx + b$ für die Plots aus \autoref{abb:Bfelf}}
  \label{tab:Fit-BFeld}
  \begin{tabular}{c c c}
    \toprule
    Isotop & $m$ / \si{\tesla\per\hertz} & $b$ / \si{\tesla} \\
    \midrule
    1 & \num{1.425(22)e-10} & \num{-2.8(13)e-06} \\
    2 & \num{2.151(22)e-10} & \num{-3.1(14)e-06} \\
    \bottomrule
  \end{tabular}
\end{table}

\subsection{Berechnung der Landéschen \gF-Faktoren und der zugehörigen Kernspins}
\label{sec:kernspins}
Zur Berechnung der Landéschen \gF-Faktoren wird
\begin{align*}
  \omega_0 &= \underbrace{g_{\symup F} \, \frac{\mu_{\symup B}}{\symup h}}_{\substack{m^{-1}}} \, B_0 \\
  \Rightarrow \ m &= \frac{\symup h}{\mu_{\symup B} \, g_{\symup F}}
\end{align*}
nach \gF umgestellt. Mit h als als Plankschen Wirkungsquantum und $m$ als Steigung
des Fits aus \autoref{tab:Fit-BFeld} ergibt sich
\begin{align}
  g_{\symup F} &= \frac{\symup h}{\mu_{\symup B} \, m} \\
  g_{\symup F_{1}} &= \num{0.501(8)} \\
  g_{\symup F_{2}} &= \num{0.3322(35)}
\end{align}

Um die Kernspins zu berechnen, muss als erstes der Faktor \gJ
aus \eqref{eqn:gJ} bestimmt werden. Aus \eqref{eqn:gF} lässt sich dann $I$
bestimmen. Mit $J, S = 0,5$, $L = 0$ (\cite{anleitung}) und \eqref{eqn:F-Vektor} ergibt sich schließlich
für die Kernspins
\begin{align}
  I_1 &= \num{1.497(30)} \\
  I_2 &= \num{2.514(31)} \, .
\end{align}
Im Vergleich mit den Literaturwerten (\cite{internetchemie}) für den Kernspin $\symup{^{85} Rb} = \frac{5}{2}$
und $\symup{^{87} Rb} = \frac{3}{2}$ lässt sich in guter Übereinstimmung annehmen, dass die
erste Resonanzstelle zu $\symup{^{87} Rb}$ und die zweite zu $\symup{^{85} Rb}$ korrespondiert.

\subsection{Berechnung des Isotopenverhältnisses}
In \autoref{abb:signalbild} sind die beiden Resonanzstellen bei einer Frequenz
von \SI{100}{\kilo\hertz} zu sehen.
\begin{figure}
  \centering
  \includegraphics[scale=1]{Auswertung/Daten/TEK0003.jpg}
  \caption{Typisches Signalbild bei einer Resonanzfrequenz von \SI{100}{\kilo\hertz}.}
  \label{abb:signalbild}
\end{figure}
Dabei ist der erste Peak der Grundzustand, in dem die Transmission gegen 0 geht.
Die beiden folgenden Peaks korrespondieren zu $\symup{^{87} Rb}$ bzw. zu $\symup{^{85} Rb}$.
Um das Verhältnis der Amplituden zu bilden, wird das Bild in \textit{GIMP} eingelesen
und die Differenz zwischen der horizontalen Linie und der beiden Peaks gebildet.
Es ergeben sich als \enquote{Tiefen} $T_{1, 2}$ der Peaks
\begin{align}
  T_1 &= \SI{71}{\px} \\
  T_2 &= \SI{153}{\px}
\end{align}
in Pixeln. Damit folgt, dass die Verteilung bei $\sim \SI{32}{\percent}$ ($\symup{^{87} Rb}$)
und $\sim \SI{68}{\percent}$ ($\symup{^{85} Rb}$) liegt. In der Natur liegen die
Isotope in dem Verhältnis $\sim \SI{72}{\percent}$ ($\symup{^{85} Rb}$) und
$\sim \SI{28}{\percent}$ ($\symup{^{87} Rb}$) vor \cite{internetchemie}.

\subsection{Abschätzung des quadratischen Zeeman-Effekts}
Zur Abschätzung des quadratischen Zeeman-Effekts wird \eqref{eqn:quadratischer-zeeman}
ausgerechnet. Dabei wird das maximale gemessene magnetische Feld verwendet
(\SI{140.05}{\micro\tesla} für die erste Resonanzstelle und \SI{211.83}{\micro\tesla}
für die zweite). Mit $m_F = 3$ für $\symup{^{85} Rb}$ und $m_F = 2$ für $\symup{^{87} Rb}$
und den entsprechenden Hyperfeinstrukturaufspaltungen aus \cite{anleitung} ergeben sich
\begin{align}
  U_{\symup{HF}_{1}} &= \SI{4.06(6)e-09}{\eV} \\
  U_{\symup{HF}_{2}} &= \SI{4.07(4)e-09}{\eV} \, .
\end{align}

\subsection{Auswertung der Periodendauer}
In \autoref{abb:perioden} sind die gemessenen Periodendauern gegen die RF-Amplitude
aufgetragen. Es wurde ein Fit der Form $f(x) = a + \frac{b}{x-c}$ angefertigt.
Dabei ist zu beachten, dass die beiden ersten Messpunkte beider Messreihen nicht
in den Fit oder den Plot eingeflossen sind, da im Falle der ersten Periode gar keine
Schwingung zu erkennen war und in der zweiten Periode der Wert unter zu großer
Ungenauigkeit aufgenommen wurde.
\FloatBarrier
\begin{figure}
  \centering
  \includegraphics[scale=0.7]{Auswertung/Plots/Perioden.pdf}
  \caption{Periodendauer gegen die RF-Amplitude aufgetragen. Zusätzlich ist ein
  hyperbolischer Fit eingezeichnet.}
  \label{abb:perioden}
\end{figure}
\FloatBarrier
Die Parameter des Fits sind in \autoref{tab:parameter-periode} dargestellt.
\begin{table}
  \centering
  \caption{Parameter des hyperbolischen Fits}
  \label{tab:parameter-periode}
  \begin{tabular}{c c c c}
    \toprule
    Isotop & a / \si{\milli\second} & b / \si{\volt\milli\second} & c / \si{\volt} \\
    \midrule
    $\symup{^{87} Rb}$ & \num{0.03(5)} & \num{1.93(34)} & \num{0.16(21)} \\
    $\symup{^{85} Rb}$ & \num{0.062(19)} & \num{3.09(13)} & \num{0.20(5)} \\
    \bottomrule
  \end{tabular}
\end{table}
Für den Quotienten $b(\symup{^{87} Rb})/b(\symup{^{85} Rb})$ ergibt sich $\num{0.62(11)}$.
In \autoref{sub-abb:87} und \autoref{sub-abb:85} sind die ansteigenden Flanken der
Periodendauern zu sehen.

\begin{figure}
  \centering
  \begin{subfigure}{0.46\textwidth}
    \centering
    \includegraphics[width=\textwidth]{Auswertung/Daten/TEK0004.jpg}
    \caption{Ansteigende Flanke der Periodendauer für $\symup{^{87} Rb}$.}
    \label{sub-abb:87}
  \end{subfigure}
  \qquad
  \begin{subfigure}{0.46\textwidth}
    \centering
    \includegraphics[width=\textwidth]{Auswertung/Daten/TEK0000.jpg}
    \caption{Ansteigende Flanke der Periodendauer für $\symup{^{85} Rb}$.}
    \label{sub-abb:85}
  \end{subfigure}
  \caption{Ansteigende Flanken der Periodendauern für beide Isotope.}
  \label{abb:ansteigende-flanken}
\end{figure}

\section{Diskussion}
Zur Messung des Erdmagnetfeldes lässt sich sagen, dass die gefitteten Werte
um eine Größenordnung zu klein sind. An sich wird aus \autoref{abb:Bfelf} ersichtlich,
dass der Fit gut funktioniert hat. Möglicherweise hat ein anderes, stärkeres Magnetfeld,
z.B. vom Faraday-Versuch, einen Einfluss auf diese Messung gehabt.

Dass der Fit gut funktioniert hat, zeigt sich auch an der Bestimmung der Landéschen
\gF-Faktoren und den Kernspins, die aus der Steigung der Geraden aus
\autoref{abb:Bfelf} berechnet werden. Der Vergleich mit den Literaturwerten
zeigt, dass sich Diese in der Fehlertoleranz der errechneten Werte befinden.

Das aus dem Amplitudenverhältnis berechnete Isotopenverhältnis weicht von dem Natur
gegebenen Verhältnis ab. Der Anteil an $\symup{^{85} Rb}$ ist in der Natur um
etwa vier Prozentpunkte höher, demenstprechend ist der Anteil an $\symup{^{87} Rb}$
um etwa vier Prozentpunkte zu hoch.

Die Abschätzung des quadratischen Zeemaneffektes bei den hier verwendeten Magnetfeldern
hat Werte von ungefähr \SI{e-9}{\eV} ergeben. Dies war zu erwarten, da die in diesem
Versuch verwendeten externen Magnetfelder nicht ausreichen, um die Spin-Bahn-Kopplung aufzubrechen.
Deshalb liegen die Energien des Effekts in einem niedrigen Bereich.

Zu den Periodendauern lässt sich sagen, dass die Messwerte gut zu dem Fit in \autoref{abb:perioden}
passen. Allerdings passt der Quotient nicht zum theoretischen Wert von 1,5. Allerdings
liegt der Kehrwert bei \num{1.60(29)}, enthält also in seiner Fehlertoleranz den
theoretischen Wert. Aufgrund der guten Übereinstimmung mit den theoretischen Kernspins
wurde die erste Resonanzstelle $\symup{^{87} Rb}$ und die zweite  $\symup{^{85} Rb}$
zugeordnet. Der hier berechnete Wert spricht allerdings genau für den umgekehrten
Fall. Die Messung der Periodendauern unterlag aber größeren Schwankungen als
der Messung des gesamten Horizontalfeldes, deswegen wird die Zuordnung wie sie in
\autoref{sec:kernspins} getroffen wurde, beibehalten. Durch eine wiederholte Messung
der Periodendauern in möglicherweise kleineren Schrittweiten könnte der Hyperbel-Fit
verbessert und somit das Verhältnis $b(\symup{^{87} Rb})/b(\symup{^{85} Rb})$ besser dem
theoretischen Wert angepasst werden.
