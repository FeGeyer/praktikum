\section{Auswertung}
\subsection{Bestimmung der longitudinalen Relaxationszeit}
Zur Bestimmung der longitudinalen Relaxationszeit wird \eqref{eqn:t1} genutzt.
Die gemessenen Werte sind in \autoref{tab:t1} aufgetragen.
Ein Fit der Form \eqref{eqn:t1} ergibt die Parameter
\begin{align}
  M_0 &= \SI{668(10)}{\mV} \label{eqn:M0_T1} \\
  T_1 &= \SI{2.13(7)}{\s} \label{eqn:T1} \, .
\end{align}

\begin{table}
  \centering
  \caption{Messwerte zur Bestimmung von $T_1$}
  \label{tab:t1}
  \begin{tabular}{c c}
    \toprule
    $\tau$ / \si{\ms} & $M_z(t)$ / \si{\mV} \\
    \midrule
    5.0 & -667.5 \\
    10.0 & -727.5 \\
    25.0 & -690.0 \\
    75.0 & -625.0 \\
    100.0 & -590.0 \\
    250.0 & -505.0 \\
    500.0 & -355.0 \\
    750.0 & -245.0 \\
    1000.0 & -162.5 \\
    1500.0 & 30.0 \\
    2500.0 & 192.5 \\
    7500.0 & 625.0 \\
    8000.0 & 647.5 \\
    9000.0 & 625.0 \\
    9500.0 & 642.0 \\
    \bottomrule
  \end{tabular}
\end{table}
Der Plot mit den Messwerten und dem Fit ist in \autoref{fig:T1} dargestellt.

\begin{figure}
  \centering
  \includegraphics[scale=0.7]{Auswertung/Plots/T1.pdf}
  \caption{Grafische Darstellung der Werte mit einem Fit der Form
  \eqref{eqn:t1}. Dabei ist $M_0$ der aus dem Fit bestimmte Parameter.}
  \label{fig:T1}
\end{figure}

\subsection{Bestimmung der transversalen Relaxationszeit}
Im Laufe der Messung wurden Burstsequenzen nach der Carr-Purcell-Methode und
nach der Meiboom-Gill-Methode aufgenommen. Die jeweiligen Sequenzen sind in
\autoref{subfig:Carr-Purcell} und in \autoref{subfig:Meiboom-Gill} zu sehen.

\begin{figure}
  \centering
  \begin{subfigure}{0.469\textwidth}
    \centering
    \includegraphics[width=\textwidth]{Auswertung/Plots/CP.pdf}
    \caption{Burstsequenz nach der Carr-Purcell-Methode.}
    \label{subfig:Carr-Purcell}
  \end{subfigure}
  \qquad
  \begin{subfigure}{0.469\textwidth}
    \centering
    \includegraphics[width=\textwidth]{Auswertung/Plots/MG.pdf}
    \caption{Burstsequenz nach der Meiboom-Gill-Methode.}
    \label{subfig:Meiboom-Gill}
  \end{subfigure}
  \caption{Während der Versuchs aufgenommene Burstsequenzen.}
  \label{fig:Burstsequenzen}
\end{figure}

Zur Bestimmung der transversalen Relaxationszeit wird die Meiboom-Gill-Methode
verwendet. Dazu werden die Peaks aus \autoref{subfig:Meiboom-Gill} mit der
Funktion $\textsc{find\_peaks}$ aus dem Paket $\textsc{scipy.signal}$ gefiltert
und ein Fit nach \eqref{eqn:t2} durchgeführt. Der Plot dazu ist in
\autoref{fig:T2} zu sehen. Die Parameter ergeben sich zu
\begin{align}
  M_0 &= \SI{590.3(21)}{\mV} \label{eqn:M0_T2} \\
  T_2 &= \SI{1.600(10)}{\s} \label{eqn:T2}
\end{align}

\begin{figure}
  \centering
  \includegraphics[scale=0.7]{Auswertung/Plots/Peaks.pdf}
  \caption{Peaks aus \autoref{subfig:Meiboom-Gill},
  gefittet mit einer Exponentialfunktion nach \eqref{eqn:t2}.}
  \label{fig:T2}
\end{figure}

\clearpage
\subsection{Bestimmung des Diffusionskoeffizienten}
Um den Diffusionskoeffizienten zu bestimmen, muss der Feldgradient $G$ bekannt
sein. Dieser errechnet sich aus
\begin{equation}
  G = \frac{4 \cdot 2,2}{d \, \gamma \, t_{1/2}}
  \label{eqn:Feldgradient}
\end{equation}
mit $d = \SI{4.4}{\milli\meter}$, dem gyromagnetischen Faktor $\gamma =
\SI{42.576}{\mega\hertz\per\tesla} $\cite{gyro} und der Halbwertszeit $t_{1/2}$.
Während der Messung wurden für die verschiedenen Spin-Echos die x-Werte der
halben Höhe der Peaks aufgenommen und sind in \autoref{tab:halbwertsbreite}
dargestellt. Die Differenz der x-Werte ist die Halbwertsbreite, welche auch in
\autoref{tab:halbwertsbreite} zu sehen ist.

\begin{table}
  \centering
  \caption{x-Werte der halben Höhe der Peaks und die daraus errechnete
  Halbwertsbreite $t_{1/2}$}
  \label{tab:halbwertsbreite}
  \begin{tabular}{c c c}
    \toprule
    $x_1$ / \si{\ms} & $x_2$ / \si{\ms} & $t_{1/2}$ / \si{\ms} \\
    \midrule
    10.058 & 9.966 & 0.092 \\
    12.058 & 11.956 & 0.102 \\
    14.054 & 13.966 & 0.088 \\
    15.858 & 15.771 & 0.087 \\
    18.054 & 17.965 & 0.089 \\
    19.968 & 20.060 & 0.091 \\
    21.979 & 22.057 & 0.078 \\
    23.970 & 24.041 & 0.071 \\
    25.980 & 26.051 & 0.071 \\
    27.974 & 28.039 & 0.065 \\
    29.987 & 30.046 & 0.059 \\
    31.983 & 32.040 & 0.057 \\
    33.986 & 34.040 & 0.054 \\
    \bottomrule
  \end{tabular}
\end{table}

Um schließlich die Diffusionskonstante zu bestimmen, wird die Funktion
\begin{equation}
  \ln(M_y(t)) = \ln(M_0) - \frac{t}{T_2} - D \, \gamma^2 \, G^2 \, \frac{t^3}{12}
  \label{eqn:Diffusionskonstante}
\end{equation}
gefittet an die Messwerte aus \autoref{tab:D} ($t = 2\tau$). Mit dem bekannten
gyromagnetischen Faktor und dem errechneten Feldgradienten $G$ aus
\eqref{eqn:Feldgradient} ergeben sich als Parameter des Fits
\begin{align}
  M_0 &= \SI{214(11)}{\mV} \label{eqn:M0_D} \\
  D &= \SI{8.7(5)e-10}{\meter\squared\per\second} \, .
\end{align}
Die grafische Darstellung der Messwerte und des Fits befindet sich in
\autoref{fig:D}.


\begin{table}
  \centering
  \caption{Messwerte der Verzögerungszeit und der Tiefe der Peaks bei maximalem
  Feldgradienten}
  \label{tab:D}
  \begin{tabular}{c c}
    \toprule
    $\tau$ / \si{\ms} & $M_y(t)$ / \si{\mV} \\
    \midrule
    5.0 & -240.0 \\
    6.0 & -220.0 \\
    7.0 & -190.0 \\
    8.0 & -175.0 \\
    9.0 & -157.5 \\
    10.0 & -140.0 \\
    11.0 & -110.0 \\
    12.0 & -100.0 \\
    13.0 & -77.5 \\
    14.0 & -65.0 \\
    15.0 & -52.5 \\
    16.0 & -45.0 \\
    17.0 & -37.5 \\
    \bottomrule
  \end{tabular}
\end{table}

\begin{figure}
  \centering
  \includegraphics[scale=0.7]{Auswertung/Plots/Diffusion.pdf}
  \caption{Grafische Darstellung der Messwerte und des Fits.}
  \label{fig:D}
\end{figure}

Aus der Diffusionskonstante soll nach
\begin{equation}
  D = \frac{k_B \, T}{6 \, \pi \, r \, \eta}
  \label{eqn:Molekülradius}
\end{equation}
der Molekülradius $r$ bestimmt werden. Dazu wird die Viskosität $\eta$ benötigt.
Diese bestimmt sich nach
\begin{equation}
  \eta(T) = \rho \, \alpha \, (t - \delta) \, ,
\end{equation}
mit der Dichte $\rho$, einer Eichkonstante $\alpha =
\SI{1.024e-9}{\meter\squared\per\second\second}$ und $\delta$ aus einem
Zusammenhang mit der Zeit $t$, der in \autoref{tab:viskosität} aufgetragen ist.

\begin{table}
  \centering
  \caption{Zusammenhang zwischen $\delta$ und $t$}
  \label{tab:viskosität}
  \begin{tabular}{c c}
    \toprule
    $t$ / \si{\s} & $\delta$ / \si{\s} \\
    \midrule
    350.0 & 3.4 \\
    400.0 & 2.6 \\
    450.0 & 2.1 \\
    500.0 & 1.7 \\
    600.0 & 1.2 \\
    700.0 & 0.9 \\
    800.0 & 0.7 \\
    900.0 & 0.5 \\
    1000.0 & 0.4 \\
    \bottomrule
  \end{tabular}
\end{table}

Mit einer gemessenen Zeit von \SI{935.7}{\s} ergibt sich aus einem linearen Fit
mit den letzten vier Werten aus \autoref{tab:viskosität} für $\delta =
\SI{0.479}{\s}$. Damit ergibt sich eine Viskosität von
\SI{9.577e-07}{\gram\per\meter} für $\rho = 1$.

\section{Diskussion}
