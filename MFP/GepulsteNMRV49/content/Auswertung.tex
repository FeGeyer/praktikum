\section{Auswertung}
\subsection{Bestimmung der longitudinalen Relaxationszeit}
Zur Bestimmung der longitudinalen Relaxationszeit wird \eqref{eqn:t1} genutzt.
Die gemessenen Werte sind in \autoref{tab:t1} aufgetragen.
Ein Fit der Form \eqref{eqn:t1} ergibt die Parameter
\begin{align}
  M_0 &= \SI{668(10)}{\mV} \label{eqn:M0_T1} \\
  T_1 &= \SI{2.13(7)}{\s} \label{eqn:T1} \, .
\end{align}

\begin{table}
  \centering
  \caption{Messwerte zur Bestimmung von $T_1$}
  \label{tab:t1}
  \begin{tabular}{c c}
    \toprule
    $\tau$ / \si{\ms} & $M_z(t)$ / \si{\mV} \\
    \midrule
    5.0 & -667.5 \\
    10.0 & -727.5 \\
    25.0 & -690.0 \\
    75.0 & -625.0 \\
    100.0 & -590.0 \\
    250.0 & -505.0 \\
    500.0 & -355.0 \\
    750.0 & -245.0 \\
    1000.0 & -162.5 \\
    1500.0 & 30.0 \\
    2500.0 & 192.5 \\
    7500.0 & 625.0 \\
    8000.0 & 647.5 \\
    9000.0 & 625.0 \\
    9500.0 & 642.0 \\
    \bottomrule
  \end{tabular}
\end{table}
Der Plot mit den Messwerten und dem Fit ist in \autoref{fig:T1} dargestellt.

\begin{figure}
  \centering
  \includegraphics[scale=0.7]{Auswertung/Plots/T1.pdf}
  \caption{Grafische Darstellung der Werte mit einem Fit der Form \eqref{eqn:t1}.
  Dabei ist $M_0$ der aus dem Fit bestimmte Parameter.}
  \label{fig:T1}
\end{figure}

\subsection{Bestimmung der transversalen Relaxationszeit}
Im Laufe der Messung wurden Burstsequenzen nach der Carr-Purcell-Methode und nach
der Meiboom-Gill-Methode aufgenommen. Die jeweiligen Sequenzen sind in \autoref{subfig:Carr-Purcell}
und in \autoref{subfig:Meiboom-Gill} zu sehen.

\begin{figure}
  \centering
  \begin{subfigure}{0.469\textwidth}
    \centering
    \includegraphics[width=\textwidth]{Auswertung/Plots/CP.pdf}
    \caption{Burstsequenz nach der Carr-Purcell-Methode.}
    \label{subfig:Carr-Purcell}
  \end{subfigure}
  \qquad
  \begin{subfigure}{0.469\textwidth}
    \centering
    \includegraphics[width=\textwidth]{Auswertung/Plots/MG.pdf}
    \caption{Burstsequenz nach der Meiboom-Gill-Methode.}
    \label{subfig:Meiboom-Gill}
  \end{subfigure}
  \caption{Während der Versuchs aufgenommene Burstsequenzen.}
  \label{fig:Burstsequenzen}
\end{figure}

Zur Bestimmung der transversalen Relaxationszeit wird die Meiboom-Gill-Methode
verwendet. Dazu werden die Peaks aus \autoref{subfig:Meiboom-Gill} mit der
Funktion $\textsc{find\_peaks}$ aus dem Paket $\textsc{scipy.signal}$ gefiltert
und ein Fit nach \eqref{eqn:t2} durchgeführt. Der Plot dazu ist in \autoref{fig:T2}
zu sehen.

\begin{figure}
  \centering
  \includegraphics[scale=0.7]{Auswertung/Plots/Peaks.pdf}
  \caption{Peaks aus \autoref{subfig:Meiboom-Gill}, gefittet mit einer Exponentialfunktion
  nach \eqref{eqn:t2}.}
  \label{fig:T2}
\end{figure}

\section{Diskussion}
