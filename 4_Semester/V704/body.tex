\maketitle
\tableofcontents
\newpage
\section{Theorie}
\subsection{Wirkungsquerschnitt und exponentielles Absorbtionsgesetz}

In diesem Versuch sollen das Wechselwirkungsverhalten von hochenergetischer Strahlung
mit Materie untersucht werden. Dabei werden sowohl hochenergie-Photonen (also $\gamma$-Strahlung)
als auch stark beschleunigte Elektronen zwischen \num{60} und \SI{1300}{\kilo\electronvolt}
(also $\beta^-$-Strahlung) betrachtet.\\
Für die betrachtung beider Strahlungsarten wird der Begriff des \textbf{Wirkungsquerschnitts}
benötigt. Der Wirkungsquerschnitt $\sigma$ wird nun so definiert, dass er die Fläche senkrecht
zur Einfallsrichtung der Strahlung umfasst, in der ein Teilchen des bestrahlten Materials
mit der einfallenden Strahlung wechselwirken kann. Nimmt man nun eine gleichmäßige
Verteilung der Teilchen in Schichten an, so erhält man durch Integration das \textbf{exponentielle
Absorbtionsgesetz}
\begin{equation}
  N(D) = N_0 \exp^{-\mu D}
  \label{eqn:1}
\end{equation}
mit Absorptonskoeffizient $\mu = n \sigma$ und Durchmesser der Absorberplatte $D$.
Die Konstante $n$ des Absorbtionskoeffizienten gibt dabei die Anzahl von Teilchen
pro Volumenelement des Absorbermaterials an, die Amplitude $N_0$ im Absorptionsgesetz die
Anzahl der auf die Absorberplatte pro Zeiteinheit eintreffenden Teilchen.

\subsection{Wechselwirkung von Gamma-Strahlung mit Materie}
Dieses Gesetz lässt sich gut auf $\gamma$-Strahlung anwenden, die in guter Näherung
nur eine Wehcselwirkung pro $\gamma$-Quant durchführt. Quelle der $\gamma$-Strahlung ist
hierbei das Abfallen angeregter Kernzustände auf energetisch stabilere, energieärmere
Zustände. Die dabei verlorene Energie wird großteils in Form eines $\gamma$-Quants frei.
Da die möglichen Energieniveaus eines gebundenen Elektrons diskret sind, ist auch das
mögliche Spektrum der $\gamma$-Strahlung diskret. Die möglichen Wechselwirkungen mit Materie
lassen sich dabei in Annihlationsprozesse sowie elastische- und inelastische Streuung
einteilen. Bei den $\gamma$-Energien von
\SI{10}{\kilo\electronvolt} bis \SI{10}{\mega\electronvolt} die in diesem Experiment
erreicht werden, treten vor allem die folgenden drei Arten von Wechselwirkungen auf:
\begin{enumerate}
  \item \textbf{Photo-Effekt}: Beim (inneren) Photo-Effekt, einem Annihlationseffekt,
  schlägt das einfallende $\gamma$-Quant ein Elektron aus der Hülle eines Teilchens.
  Das $\gamma$-Quant wird dabei vernichtet, das Elektron erhält eine kinetische
  Energie, die von der Energie des einfallenden $\gamma$-Teilchens ($E_\gamma = \symup{h}\nu$)
  sowie der Bindungsenergie $E_\symup{B}$ des Elektrons abhängig und durch
  \begin{equation}
    E_\symup{e} = E_\gamma - E_\symup{B}
    \label{eqn:2}
  \end{equation}
  gegeben ist. Die Bindungsenergie des Elektrons bietet daher eine untere Schranke
  für die Energien, bei denen der Photo-Effekt eintritt. Die Impulserhaltung sorgt
  weiter dafür, dass der Photo-Effekt in den inneren, stärker gebunden Elektronenschalen
  wahrscheinlicher ist. Außerdem steigt die Wahrscheinlichkeit aus dem selben Grund
  für schwere Atome, da die Bindungsenergie hier stärker ist.
  \item \textbf{Compton-Effekt}: Als Compton-Effekt bezeichnet man die inelastische Streuung
  von $\gamma$-Quanten an freien Elektronen. Bei hohen Energien kommen dafür auch die äußeren
  Hüllenelektronen in Frage. Das $\gamma$-Quant wird dabei nicht vernichtete, es verliert einen
  Teil seiner Energie und beschleunigt dabei das getroffene Elektron. Beide Teilchen bewegen
  sich dabei weiter, ändern ihre Richtung jedoch (sie werden "gestreut"). Dadurch nimmt
  die Intensität des Teilchenstrahls ab. Der dabei zu betrachtene Wirkungsquerschnitt bestimmt
  sich jedoch nicht nach dem weiter oben genannten Zusammenhang. Für $\gamma$-Quanten, deren
  Energie klein im Vergleich zur Ruheenergie des getroffenen Elektrons ist. $\sigma$ bestimmt sich
  dann näherungsweise als
  \begin{equation}
    \sigma = \frac{8}{3} \pi \symup{r}_\symup{e}^2
  \end{equation}
  mit dem klassischen Elektronenradius $\symup{r}_\symup{e}$.
  \item \textbf{Paarbildung}: Für sehr große $\gamma$-Energien oberhalb der doppelten
  Ruhemasse der Elektronen, wird das $\gamma$-Quant unter Bildung eines Elektron-Positron-Paares
  annihliert. Der Betrag $2 \symup{m}_0 \symup{c}^2$ ist hierbei als untere Schranke
  nicht ausreichend, da ein weiterer Stoßpartner notwendig ist, der einen Teil des Impulses
  des $\gamma$-Quants aufnimmt, damit der Vorgang der Impulserhaltung genügt.
\end{enumerate}
Die oben genannten Effekte treten beim Durchgang eines $\gamma$-Strahls durch ein
Absorbermaterial zusammen auf und überlagern sich daher, es kann jedoch in bestimmten
Energiebereichen eine Dominanz bestimmter Effekte beobachtet werden. Der Verlauf einer solchen
Kurve ist in Abbildung \ref{abb:1} dargestellt.
\subsection{Wechselwirkung von Beta-Strahlung mit Materie}

\section{Durchführung}
\subsection{Versuchsaufbau}
\subsection{Versuchsdurchführung}
\section{Auswertung}
\section{Diskussion}
\newpage
\nocite{*}
\printbibliography
