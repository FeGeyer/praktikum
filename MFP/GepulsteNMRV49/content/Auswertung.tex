\section{Auswertung}
\subsection{Eingestellte Werte}
Für die folgenden Versuchsteile wurden die Shim-Parameter auf die in
\autoref{tab:shim} dargestellten Werte eingestellt.

\begin{table}
  \centering
  \caption{Verwendetet Werte für die Shim-Parameter}
  \label{tab:shim}
  \begin{tabular}{c c c c}
    \toprule
    x & y & z & $\symup{z^2}$ \\
    \midrule
    -1.67 & -5.6 & +3.37 & -2.15 \\
    \bottomrule
  \end{tabular}
\end{table}

Für die Pulsparameter wurden die Werte in \autoref{tab:parameter-pulse}
eingestellt.

\begin{table}
  \caption{Frequenz, Phase, Länge der beiden Pulse, Pulsabstand, Anzahl der zweiten
  Pulse und Periodenlänge}
  \label{tab:parameter-pulse}
  \begin{tabular}{c c c c c c c}
    \toprule
    f / \si{\mega\hertz} & $\phi$ / \si{\deg} & A / \si{\micro\second} &
    B / \si{\micro\second} & $\tau$ / \si{\s} & N & P / \si{\s} \\
    \midrule
    \num{21.71} & -107 & 4.8 & 9.6 & 0.0129 & 100 & 12 \\
    \bottomrule
  \end{tabular}
\end{table}

Falls die Werte abweichen, wird das am Anfang des Unterkapitels erwähnt.
\subsection{Bestimmung der longitudinalen Relaxationszeit}
Diese Messung wird bei einer Periodenlänge von \SI{22}{\second} durchgeführt.
Zur Bestimmung der longitudinalen Relaxationszeit wird \eqref{eqn:t1} genutzt.
Die gemessenen Werte sind in \autoref{tab:t1} aufgetragen.
Ein Fit der Form \eqref{eqn:t1} ergibt die Parameter
\begin{align}
  M_0 &= \SI{668(10)}{\mV} \label{eqn:M0_T1} \\
  T_1 &= \SI{2.13(7)}{\s} \label{eqn:T1} \, .
\end{align}

\begin{table}
  \centering
  \caption{Messwerte zur Bestimmung von $T_1$}
  \label{tab:t1}
  \begin{tabular}{c c}
    \toprule
    $\tau$ / \si{\ms} & $M_z(t)$ / \si{\mV} \\
    \midrule
    5.0 & -667.5 \\
    10.0 & -727.5 \\
    25.0 & -690.0 \\
    75.0 & -625.0 \\
    100.0 & -590.0 \\
    250.0 & -505.0 \\
    500.0 & -355.0 \\
    750.0 & -245.0 \\
    1000.0 & -162.5 \\
    1500.0 & 30.0 \\
    2500.0 & 192.5 \\
    7500.0 & 625.0 \\
    8000.0 & 647.5 \\
    9000.0 & 625.0 \\
    9500.0 & 642.0 \\
    \bottomrule
  \end{tabular}
\end{table}
Der Plot mit den Messwerten und dem Fit ist in \autoref{fig:T1} dargestellt.

\begin{figure}
  \centering
  \includegraphics[scale=0.7]{Auswertung/Plots/T1.pdf}
  \caption{Grafische Darstellung der Werte mit einem Fit der Form
  \eqref{eqn:t1}. Dabei ist $M_0$ der aus dem Fit bestimmte Parameter.}
  \label{fig:T1}
\end{figure}

\subsection{Bestimmung der transversalen Relaxationszeit}
Im Laufe der Messung wurden Burstsequenzen nach der Carr-Purcell-Methode und
nach der Meiboom-Gill-Methode aufgenommen. Die jeweiligen Sequenzen sind in
\autoref{subfig:Carr-Purcell} und in \autoref{subfig:Meiboom-Gill} zu sehen.

\begin{figure}
  \centering
  \begin{subfigure}{0.469\textwidth}
    \centering
    \includegraphics[width=\textwidth]{Auswertung/Plots/CP.pdf}
    \caption{Burstsequenz nach der Carr-Purcell-Methode.}
    \label{subfig:Carr-Purcell}
  \end{subfigure}
  \qquad
  \begin{subfigure}{0.469\textwidth}
    \centering
    \includegraphics[width=\textwidth]{Auswertung/Plots/MG.pdf}
    \caption{Burstsequenz nach der Meiboom-Gill-Methode.}
    \label{subfig:Meiboom-Gill}
  \end{subfigure}
  \caption{Während der Versuchs aufgenommene Burstsequenzen.}
  \label{fig:Burstsequenzen}
\end{figure}

Zur Bestimmung der transversalen Relaxationszeit wird die Meiboom-Gill-Methode
verwendet. Dazu werden die Peaks aus \autoref{subfig:Meiboom-Gill} mit der
Funktion $\textsc{find\_peaks}$ aus dem Paket $\textsc{scipy.signal}$ gefiltert
und ein Fit nach \eqref{eqn:t2} durchgeführt. Der Plot dazu ist in
\autoref{fig:T2} zu sehen. Die Parameter ergeben sich zu
\begin{align}
  M_0 &= \SI{590.3(21)}{\mV} \label{eqn:M0_T2} \\
  T_2 &= \SI{1.600(10)}{\s} \label{eqn:T2}
\end{align}

\begin{figure}
  \centering
  \includegraphics[scale=0.7]{Auswertung/Plots/Peaks.pdf}
  \caption{Peaks aus \autoref{subfig:Meiboom-Gill},
  gefittet mit einer Exponentialfunktion nach \eqref{eqn:t2}.}
  \label{fig:T2}
\end{figure}

\clearpage
\subsection{Bestimmung des Diffusionskoeffizienten}
Zu Beginn dieses Versuchsteils wird z auf -10 und N auf 1 gesetzt.
Um den Diffusionskoeffizienten zu bestimmen, muss der Feldgradient $G$ bekannt
sein. Dieser errechnet sich aus
\begin{equation}
  G = \frac{4 \cdot 2,2}{d \, \gamma \, t_{1/2}}
  \label{eqn:Feldgradient}
\end{equation}
mit $d = \SI{4.4}{\milli\meter}$, dem gyromagnetischen Faktor $\gamma =
\SI{42.576}{\mega\hertz\per\tesla} $\cite{gyro} und der Halbwertszeit $t_{1/2}$.
Während der Messung wurden für die verschiedenen Spin-Echos die x-Werte der
halben Höhe der Peaks aufgenommen und sind in \autoref{tab:halbwertsbreite}
dargestellt. Die Differenz der x-Werte ist die Halbwertsbreite, welche auch in
\autoref{tab:halbwertsbreite} zu sehen ist. Zur Berechnung wurde der Mittelwert
der Differenzen gebildet, dieser ergibt sich zu
$t_{1/2} = \SI{77.27(431)}{\micro\second}$.

\begin{table}
  \centering
  \caption{x-Werte der halben Höhe der Peaks und die daraus errechnete
  Halbwertsbreite $t_{1/2}$}
  \label{tab:halbwertsbreite}
  \begin{tabular}{c c c}
    \toprule
    $x_1$ / \si{\ms} & $x_2$ / \si{\ms} & $t_{1/2}$ / \si{\ms} \\
    \midrule
    10.058 & 9.966 & 0.092 \\
    12.058 & 11.956 & 0.102 \\
    14.054 & 13.966 & 0.088 \\
    15.858 & 15.771 & 0.087 \\
    18.054 & 17.965 & 0.089 \\
    19.968 & 20.060 & 0.091 \\
    21.979 & 22.057 & 0.078 \\
    23.970 & 24.041 & 0.071 \\
    25.980 & 26.051 & 0.071 \\
    27.974 & 28.039 & 0.065 \\
    29.987 & 30.046 & 0.059 \\
    31.983 & 32.040 & 0.057 \\
    33.986 & 34.040 & 0.054 \\
    \bottomrule
  \end{tabular}
\end{table}

Um schließlich die Diffusionskonstante zu bestimmen, wird die Funktion
\begin{equation}
  \ln(M_y(t)) = \ln(M_0) - \frac{t}{T_2} - D \, \gamma^2 \, G^2 \, \frac{t^3}{12}
  \label{eqn:Diffusionskonstante}
\end{equation}
gefittet an die Messwerte aus \autoref{tab:D} ($t = 2\tau$). Mit dem bekannten
gyromagnetischen Faktor und dem errechneten Feldgradienten $G$ aus
\eqref{eqn:Feldgradient} ergeben sich als Parameter des Fits
\begin{align}
  M_0 &= \SI{214(11)}{\mV} \label{eqn:M0_D} \\
  D &= \SI{8.7(5)e-10}{\meter\squared\per\second} \, .
\end{align}
Die grafische Darstellung der Messwerte und des Fits befindet sich in
\autoref{fig:D}.


\begin{table}
  \centering
  \caption{Messwerte der Verzögerungszeit und der Tiefe der Peaks bei maximalem
  Feldgradienten}
  \label{tab:D}
  \begin{tabular}{c c}
    \toprule
    $\tau$ / \si{\ms} & $M_y(t)$ / \si{\mV} \\
    \midrule
    5.0 & -240.0 \\
    6.0 & -220.0 \\
    7.0 & -190.0 \\
    8.0 & -175.0 \\
    9.0 & -157.5 \\
    10.0 & -140.0 \\
    11.0 & -110.0 \\
    12.0 & -100.0 \\
    13.0 & -77.5 \\
    14.0 & -65.0 \\
    15.0 & -52.5 \\
    16.0 & -45.0 \\
    17.0 & -37.5 \\
    \bottomrule
  \end{tabular}
\end{table}

\begin{figure}
  \centering
  \includegraphics[scale=0.7]{Auswertung/Plots/Diffusion.pdf}
  \caption{Grafische Darstellung der Messwerte und des Fits.}
  \label{fig:D}
\end{figure}

Aus der Diffusionskonstante soll nach
\begin{equation}
  D = \frac{k_B \, T}{6 \, \pi \, r \, \eta}
  \label{eqn:Molekülradius}
\end{equation}
der Molekülradius $r$ bestimmt werden. Dazu wird die Viskosität $\eta$ benötigt.
Diese bestimmt sich nach
\begin{equation}
  \eta(T) = \rho \, \alpha \, (t - \delta) \, ,
\end{equation}
mit der Dichte $\rho$, einer Eichkonstante $\alpha =
\SI{1.024e-9}{\meter\squared\per\second\second}$ und $\delta$ aus einem
Zusammenhang mit der Zeit $t$, der in \autoref{tab:viskosität} aufgetragen ist.

\begin{table}
  \centering
  \caption{Zusammenhang zwischen $\delta$ und $t$}
  \label{tab:viskosität}
  \begin{tabular}{c c}
    \toprule
    $t$ / \si{\s} & $\delta$ / \si{\s} \\
    \midrule
    350.0 & 3.4 \\
    400.0 & 2.6 \\
    450.0 & 2.1 \\
    500.0 & 1.7 \\
    600.0 & 1.2 \\
    700.0 & 0.9 \\
    800.0 & 0.7 \\
    900.0 & 0.5 \\
    1000.0 & 0.4 \\
    \bottomrule
  \end{tabular}
\end{table}

Mit einer gemessenen Zeit von \SI{935.7}{\s} ergibt sich aus einem linearen Fit
mit den letzten vier Werten aus \autoref{tab:viskosität} für $\delta =
\SI{0.479}{\s}$. Damit ergibt sich eine Viskosität von
\SI{0.000955942187783885}{\kilo\gram\per\meter\per\second} für
$\rho = \SI{998.2}{\kilo\gram\per\cubic\meter}$ bei $T = \SI{293.15}{\K}$.
Mit diesem Wert ergibt sich für den Molekülradius aus \eqref{eqn:Molekülradius}
ein Wert von $r = \SI{2.58(14)e-10}{\meter}$.

Um diesen Wert mit theoretischen Berechnungen zu vergleichen werden nach
\begin{equation}
  \rho \cdot 0.74 = \frac{m_{\symup{H_2O}}}{\frac{4}{3}\pi r^3}
  \label{eqn:hexagonal}
\end{equation}
und
\begin{equation}
  r_{\symup{VdW}} = \left(\frac{3 k_B T_k}{128 \pi P_k} \right)^{\frac{1}{3}}
  \label{eqn:kritisch}
\end{equation}
Werte für den Molekülradius berechnet. \eqref{eqn:hexagonal} stammt aus der
Annahme, dass die Moleküle der Flüssigkeit eine hexagonal dichteste Kugelpackung
einnehmen; \eqref{eqn:kritisch} stammt aus der Annahme, dass es sich um ein
Van-der-Waals Gas am kritischen Punkt handelt. Es ergeben sich mit
$m_{\symup{H_2O}} = \SI{28.89e-27}{\kg}$, $T_k = \SI{647.05}{\K}$ \cite{kritisch},
$P_k = \SI{22.04e6}{\pascal}$ \cite{kritisch} und der bekannten Dichte:
\begin{align}
  r_{\symup{hcp}} &= \SI{2.1057e-10}{\m} \label{eqn:ergebnis_hexagonal} \\
  r_{\symup{VdW}} &= \SI{1.4461e-10}{\m} \label{eqn:ergebnis_vdw}
\end{align}

\section{Diskussion}
\subsection{Bestimmung der longitudinalen und transversalen Relaxationszeit}
Zur Bestimmung der Relaxationszeiten lässt sich sagen, dass die Messung gut
verlaufen ist. Es gilt nach \eqref{eqn:T1} und \eqref{eqn:T2} $T_1 > T_2$, was
zu erwarten war. Außerdem legen die guten Übereinstimmungen der Fits aus
\eqref{fig:T1} und \eqref{fig:T2} nahe, dass die Messung gut verlaufen ist.

\subsection{Bestimmung des Diffusionskoeffizienten}
Die Bestimmung des Diffusionskoeffizienten ist eher schlecht verlaufen. In
\autoref{fig:D} wird ersichtlich, dass der Fit eher schlecht zu den gegebenen
Daten passt. Das liegt höchstwahrscheinlich daran, dass das gewählte Intervall
der Verzögerungszeiten zu klein ist. Aus Vergleichen mit den Messungen anderer
Gruppen wird ersichtlich, dass diese kleinere minimale und größere maximale
Werte aufgenommen haben. Zumindest an der oberen Grenze war dies in unserem Fall
nicht unbedingt möglich, da sich die Peaks ab \SI{18}{\ms} nicht mehr vom
Rauschen unterschieden haben. Hier wäre eine wiederholte Messung mit angepasster
Justage und mehr aufgenommenen Messwerten eine Möglichkeit, um den Wert zu
verbessern.

\subsection{Bestimmung des Molekülradius}
Die Bestimmung des Molekülradius aus der Diffusionskonstante ist gut verlaufen.
Der Wert liegt von der Größenordnung her in einem Bereich, der allgemein als
Radius für Atomkerne angesehen wird ($\sim \SI{e-10}{\m}$). Allerdings liegt
keiner der beiden berechneten Theoriewerte im Fehlerintervall des Radius. Die
Abweichungen betragen \SI{0.19(4)}{\percent} für \eqref{eqn:ergebnis_hexagonal}
und \SI{0.441(30)}{\percent} für \eqref{eqn:ergebnis_vdw}. Dies liegt
wahrscheinlich an der schon erwähnten eher schlecht verlaufenen Bestimmung des
Diffusionskoeffizienten, der direkt in die Berechnung des Molekülradius
einfließt. Außerdem ist die Bestimmung der Viskosität aus einer Messung mit
einer Stoppuhr ein weiterer Unsicherheitsfaktor, der möglicherweise durch andere
Methoden der Viskositätsbestimmung minimiert werden könnte.
