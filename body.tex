\maketitle
\tableofcontents
\newpage

\section{Zielsetzung}
In diesem Versuch geht es um Messung der Schwingungs- und Schwebungsdauer von gekoppelten Pendeln.
Untersucht werden gleich- und gegensinnige sowie gekoppelte Schwingungen.
\section{Theorie}
Ein einzelndes Fadenpendel mit einem Faden der Länge $\textit{l}$, Masse $\textit{m}$, welches reibungsfrei aufgehängt wurde,
schwingt für kleine Auslenkungen (sin $\phi \approx \phi$) mit der Schwingungsfrequenz:
\begin{equation}
  \omega = \sqrt{\frac{\textit{l}}{g}}
  \label{e1}
\end{equation}
Dies ist die Lösung der zugehörigen Schwingungsdifferentialgleichung. Mit \eqref{e1} und
\begin{equation*}
  \textit{T} = \frac{2\pi}{\omega}
\end{equation*}
ergibt sich als Formel für die gesuchte Schwingungsdauer:
\begin{equation}
  \textit{T} = 2\pi \sqrt{\frac{g}{\textit{l}}}
  \label{e2}
\end{equation}
Wenn man nun zwei dieser Fadenpendel durch eine Feder koppelt, ergeben sich zwei DGL's mit jeweils einem Term darin,
der den Drehwinkel des anderen Pendels enthält. Dies kommt durch die Kopplung mit der Feder. Je nach Auslenkungswinkel $\alpha_{1}$
und $\alpha_{2}$ der Fadenpendel ergeben sich verschiedene Schwingungsarten:
\\
\\
Für: $\alpha_{1} = \alpha_{2}$ ergibt sich eine gleichsinnige Schwingung. Bei dieser hat die Feder keine Auswirkung auf die
Schwingungen. Deshalb gilt für die Schwingungsfrequenz $\omega_{+}$ die Formel \eqref{e1} und für die Schwingungsdauer $\textit{T}_{+}$
die Formel \eqref{e2}.
\section{Aufbau und Durchführung}
  \subsection{Aufbau}
  \subsection{Durchführung}
\section{Auswertung}

\newpage
\nocite{*}
\printbibliography
