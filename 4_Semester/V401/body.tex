\maketitle
\setcounter{page}{1}
\tableofcontents
\newpage
\pagenumbering{arabic}
\section{Theorie}
\begin{equation}
  \symup{\Delta} d = z \frac{\lambda}{2}
  \label{eq:1}
\end{equation}
\begin{equation}
  b \cdot \symup{\Delta}n = \frac{z \lambda}{2}
  \label{eq:2}
\end{equation}
\begin{equation}
  n(p_0, T_0) = 1 + \symup{\Delta}n \frac{T \cdot p_0}{T_0 \cdot (p - p^\prime)}
  \label{eq:3}
\end{equation}
\section{Durchführung}
\subsection{Versuchsaufbau}
\section{Auswertung}
\subsection{Wellenlänge eines Helium-Neon-Lasers}
\begin{table}
  \centering
  \caption{Messwerte für die Messung an einem Helium-Neon-Laser sowie für jedes Wertepaar
  berechnete Wellenlänge.}
  \begin{tabular}{c c c}
    \toprule
    $\symup{\Delta}d$ / \si{\milli\metre} & $z$ & $\lambda$ \\
    \midrule
    2 & 1286 & 619.98 \\
    2 & 1327 & 600.82 \\
    2 & 1284 & 620.94 \\
    2 & 1305 & 610.95 \\
    2 & 1340 & 594.99 \\
    2 & 1288 & 619.01 \\
    2 & 1326 & 601.27 \\
    2 & 1300 & 613.30 \\
    2 & 1330 & 599.47 \\
    2 & 1292 & 617.10 \\
    \bottomrule
  \end{tabular}
  \label{tab:1}
\end{table}

Die Messwerte finden sich in Tabelle \ref{tab:1}. Ebenfalls in dieser Tabelle finden
sich die nach \eqref{eq:wel} berechneten


\subsection{Brechungsindex von Luft und CO2}

\begin{table}[h]
  \centering
  \caption{In Tabelle \subref{tab:2} sind die gemessenen Werte bei Füllung der
  Kammer mit Luft, in Tabelle \subref{tab:3} für CO2 eingetragen. Außerdem
  eingetragen ist der für jedes Wertepaar berechnete Brechungsindex $n$.}
    \begin{subtable}{0.49\textwidth}
    \centering
    \begin{tabular}{c c c}
      \toprule
      $P$ / \si{\bar} & $z$ & $n$ \\
      \midrule
      0.2132 & 46 & 1.000416 \\
      0.2132 & 33 & 1.000298 \\
      0.2132 & 47 & 1.000425 \\
      0.2132 & 32 & 1.000289 \\
      0.2132 & 37 & 1.000334 \\
      0.2132 & 35 & 1.000316 \\
      0.2132 & 33 & 1.000298 \\
      0.2132 & 50 & 1.000452 \\
      0.2132 & 34 & 1.000307 \\
      0.2132 & 33 & 1.000298 \\
      \bottomrule
    \end{tabular}
    \caption{Werte für Luft.}
    \label{tab:2}
  \end{subtable}
    \begin{subtable}{0.49\textwidth}
    \centering
    \begin{tabular}{c c c}
      \toprule
      $P$ / \si{\bar} & $z$ & $n$ \\
      \midrule
      0.2132 & 52 & 1.000470 \\
      0.2132 & 50 & 1.000452 \\
      0.2132 & 50 & 1.000452 \\
      0.2132 & 51 & 1.000461 \\
      0.2132 & 53 & 1.000479 \\
      0.2132 & 52 & 1.000470 \\
      0.2132 & 50 & 1.000452 \\
      0.2132 & 50 & 1.000452 \\
      0.2132 & 52 & 1.000470 \\
      0.2132 & 52 & 1.000470 \\
      \bottomrule
    \end{tabular}
    \caption{Werte für CO2.}
    \label{tab:3}
    \end{subtable}

\end{table}

\section{Diskussion}
\newpage
\nocite{*}
\printbibliography
