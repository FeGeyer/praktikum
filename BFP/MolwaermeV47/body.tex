\maketitle
\setcounter{page}{1}
\newpage
\pagenumbering{arabic}
\section{Theorie}
Im folgenden werden das klassische, das einsteinsche und das Debye-Modell
zur Erläuterung der Temperaturabhängigkeit von Molwärme kristalliner Festkörper
eingeführt und diskutiert.
\subsection{Die klassische Theorie}
Klassisch gesehen verteilt sich die einem Frstkörper zugeführte Wärmeenergie gleichmäßig
auf alle Freiheitsgrade der Atome. Dies nennt man das Äquipartitionstheorem. Es gilt
\begin{equation*}
  \left< E \right> = \frac{1}{2} \, \symup{k_B} \, T
\end{equation*}
für die mittlere Energie pro Freiheitsgrad mit $T$ als Temperatur und $\symup{k_B}$ als
Boltzmannkonstante. Da diese Atome im Kristall an definierten Positionen im Gitter
sitzen, können sie entlang dreier Raumrichtungen um ihre Gleichgewichtslage schwingen.
Es folgt
\begin{equation}
  U = 3 \, N_L \, \symup{k_B} \, T = 3 \, R \, T
  \label{klassisch}
\end{equation}
für die mittlere Energie pro Atom mit $N_L$ als Loschmidtsche Zahl und $R$ als
allgemeine Gaskonstante. Die spezifische Molwärme bei konstantem Volumen ergibt
sich aus \eqref{klassisch}
\begin{equation}
  C_V = \left(\frac{\partial U}{\partial T} \right)_V = 3R \, .
  \label{eqn:1}
\end{equation}
Diese ist offensichtlich weder material- noch temperaturabhängig. Dies widerspricht
experimentellen Erfahrungen. Allerdings lässt sich der Wert $3R$ asymptotisch für
hohe Temperaturen erreichen.
\subsection{Einstein-Modell}
Die fehlende Material- und Temperaturabhängigkeit rührt daher, dass im klassischen
Ansatz die Quantelung der Schwingungsenergie außer Acht gelassen wurde. Im Einstein-Modell
schwingen alle Atome mit der Frequenz $\omega$ und können nur Schwingungsenergien
aufnehmen, die
\begin{equation}
  E = n\, \hbar \, \omega \ \ \forall \, n \in \mathbb{N}_0
  \label{eqn:2}
\end{equation}
erfüllen. Weiterhin gilt für die Wahrscheinlichkeit, mit der ein schwingendes Atom
bei einer Temperatur $T$ im thermodynamischen Gleichgewicht steht und die Energie
\eqref{eqn:2} besitzt, nach der Boltmann-Verteilung
\begin{equation}
  W(n) = \symup{exp}\left(- \frac{n \, \hbar \, \omega}{k_B \, T} \right) \, .
\end{equation}
Es gilt
\begin{equation}
  \left< U \right >_\symup{Einstein} = \frac{\sum_{n = 0}^\infty n \hbar \omega \,
  \symup{exp}\left(- \frac{n \, \hbar \, \omega}{k_B \, T} \right)}
  {\sum_{n = 0}^\infty \symup{exp}\left(- \frac{n \, \hbar \, \omega}{k_B \, T} \right)}
\end{equation}
und es folgt daraus
\begin{equation}
  \left< U \right >_\symup{Einstein} = \frac{\hbar \, \omega}
  {\symup{exp}\left(\frac{\hbar \, \omega}{k_B \, T} \right) - 1} \, .
  \label{eqn:3}
\end{equation}
Für die Molwärme ergibt sich aus \eqref{eqn:3}
\begin{equation}
  {C_V}_\symup{Einstein} = \symup{\frac{d}{dT}} 3 N_L \, \frac{\hbar \, \omega}
  {\symup{exp}\left(\frac{\hbar \, \omega}{k_B \, T} \right) - 1} =
  3R \, \frac{\hbar^2 \omega^2}{k^2 T^2} \frac{\symup{exp}\left(\hbar \omega / k_B T \right)}
  {\left[\symup{exp}\left(\hbar \omega / k_B T \right) - 1 \right]^2}
  \label{eqn:4}
\end{equation}
Für \eqref{eqn:4} gilt wie erwähnt im Grenzfall großer Temperaturen
\begin{equation}
  \lim\limits_{T \to \infty}{{C_V}_\symup{Einstein}} = 3R \, .
\end{equation}
Außerdem enthält \eqref{eqn:4} die erwartete Abnahme der Molwärme mit der Temperatur.
Allerdings weicht vor allem der Verlauf im Bereich tiefer Temperatuen stark ab vom
Verlauf der experimentellen Kurve.

\section{Durchführung}
\subsection{Versuchsaufbau}
\subsection{Versuchsdurchführung}
\section{Auswertung}
\subsection{Fehlerrechnung}
Die Fehlerrechnung wird in $\textsc{Python}$\footnote{Version: 3.6.3} durchgeführt.
Mittelwerte werden durch die Funktion $\textsc{mean}$ aus dem Paket $\textsc{Numpy}$\footnote{Version: 3.6.3},
die zugehörigen Standartabweichungen durch die Funktion $\textsc{stats.sem}$ aus dem
Paket $\textsc{scipy}$\footnote{Version: 1.0.0} berechnet. Fehlerfortpflanzung wird
durch die Bibliothek $\textsc{uncertainties.unumpy}$\footnote{Version: 3.0.1} automatisiert.

\section{Diskusion}
\newpage
\nocite{*}
\printbibliography
