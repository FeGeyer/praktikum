\maketitle
\setcounter{page}{1}
\tableofcontents
\newpage
\pagenumbering{arabic}
\section{Theorie}
Ziel des Versuchs ist die Untersuchung der Temperaturabhängigkeit des glühelektrischen
Effektes an einer Hochvakuumdiode. Als glühelektrischen Effekt wird die Emission von Elektronen aus einem
aufgeheizten Metall bezeichnet. Dabei muss an den Elektronen die sogenannte
Austrittsarbeit geleistet werden. Zur Erklärung dieses Begriffs ist die Vorstellung
des Metalls als Potentialtopf sinnvoll. Die positiv geladenen Kerne sind ein einem
Gitter gebunden, das Potential der Kerne erzeugt einen Potentialtopf, der um $\phi$
vom umgebenden Potential abweicht. In diesem
Topf können sich die Elektronen frei bewegen, sie sind nicht mehr an ein Atom gebunden.
Um den Potentialtopf verlassen zu können, müssen die Elektronen mindestens dieses Potential
überwinden. Zusätzlich müssen sie die $\textsc{Fermische}$ Grenzenergie $\zeta$ besitzen,
die aus dem $\textsc{Pauli}$-Prinzip resultiert. Sie entspricht der Energie, die
die Elektronen am absoluten Nullpunkt besitzen würden. Diese Überlegung führt zusammen
mit einigen Kentnissen über das Verhalten der Elektronen in ihrem Phasenraum auf die
$\textsc{Richardson}$-Gleichung
\begin{equation}
  j_S(T) = 4 \pi \frac{e_0 m_0 k^2}{h^3} T^2 \exp\left(\frac{-e_0 \phi}{k T} \right),
\end{equation}
mit der aus der Temperatur $T$ die Sättigungsstromdichte, also den Strom der Elektronen
aus der Metalloberfläche, berechnet werden.\\
Versuchsanordnungen zur Messung dieses Zusammenhanges schließen zwangsläufig ein
Vakuum ein, um Wechselwirkungen zwischen emitierten Elektronen und Umgebung zu Verhindern.
Zusätzlich wird in ihnen ein elektrisches Feld angelegt, um die ungerichtete Elektronenemission
zu fokusieren. Dafür wird eine Anode in die Anordnung gebracht, die Emitierende Metaloberfläche
dient als Kathode. Die Elektronen bewegen sich also nach ihrer Emission auf die Anode zu.
Einen solchen Aufbau bezeichnet man als Hochvakuumdiode. Bei Nutzung einer solchen
Anordnung stellt man jedoch fest, dass der an der Anode gemessene Strom von der Spannung
der Anode abhängt und erst ab einer Sättigungsspannung $I_S$ alle Elektronen die Annode erreichen. 
\section{Durchführung}
\subsection{Versuchsaufbau}
\subsection{Versuchsdurchführung}
\section{Auswertung}
\section{Diskussion}
\newpage
\nocite{*}
\printbibliography
