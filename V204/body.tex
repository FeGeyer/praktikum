\maketitle
\tableofcontents
\newpage

\section{Zielsetzung}
In diesem Versuch geht es um die Untersuchung der Wärmeleitung von Aluminium, Edelstahl und Messing.
\section{Theorie}
Wärmeleitung ist eine von drei Methoden, um Wärme entlang eines Temperaturgefälles zu transportieren,
welches durch eine Störung des Temperaturgleichgewichts ensteht. Neben Konvektion und Wärmestrahlung
erfolgt die Wärmeleitung durch Phononen und frei bewegliche Elektronen. Durch einen Stab der Länge L und der
Querschnittsfläche A, der aus einem Material mit Dichte $\textit{\rho}$ und spezifische Wärme $\textit{c}$ besteht,
fließt in der Zeit $\textit{dt}$ bezüglich der Querschnittsfläche die Wärmemenge
\begin{equation}
  \symup dQ = - \kappa \, A \, \frac{\partial T}{\partial x} \, \symup dt .
  \label{eqn:1}
\end{equation}
Dabei ist $\kappa$ die materialabhängige Wärmeleitfähigkeit. Mit Hilfe der Wärmestromdichte
$j_{\omega}$ und Wärmeleitungsgleichung lässt sich die Temperaturwellengleichung schreiben als
\begin{equation}
  T(x, t) = T_{max} \ e^{\sqrt{\frac{\omega \rho c}{2 \kappa}} \, x} \,
  cos \left(\omega t - \sqrt{\frac{\omega \rho c}{2 \kappa}} \, x \right)
  \label{eqn:2}
\end{equation}
für einen mit Periode T abwechselnd erwärmten und abgekühlten sehr langen Stab.
Die Phasengesschwindigkeit der Welle lautet
\begin{equation}
    v = \frac{\omega}{k} = \frac{\omega}{\sqrt{\frac{\omega \rho c}{2 \kappa}}}
    = \sqrt{\frac{2 \kappa \omega}{\rho c}}
    \label{eqn:3}
\end{equation}
zur Bestimmung der Dämpfung nutzt man das Amplitudenverhältnis aus $A_{nah}$ und $A_{fern}$
an zwei Messstellen $x_{nah}$ und $x_{fern}$. Für die Wärmeleitfähigkeit erhält man also
\begin{equation}
  \kappa = \frac{\rho c (\Delta x)^2}{2 \, \Delta t \, ln (A_{nah}/{A_{fern})}}
  \label{eqn:4}
\end{equation}
mit $\Delta x$ als Abstand und $\Delta t$ als Phasendifferenz der Temperaturwelle
zwischen den beiden Messstellen.
\section{Durchführung}
\section{Auswertung}
\section{Diskussion}
\newpage
\nocite{*}
\printbibliography
