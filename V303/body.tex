\maketitle
\tableofcontents
\newpage

\section{Zielsetzung}
Gegenstand des Versuches ist der Lock-In-Verstärker, dessen Funktionsweise
erforscht werden soll.

\section{Theorie}
\label{sec:2}
Der Lock-In-Verstärker wird überwiegend bei Messungen mit stark verrauschten Signalen
verwendet, um die gesuchte Frequenz herauszufiltern, ähnlich wie mit einem Bandpass.
Die Güte des Lock-In-Verstärkers liegt allerdings etwa um den Faktor 100 höher als
der eines Bandpasses.

Die Funktionsweise eines Lock-In-Verstärkers liegt darin,
die Messsignal mit einer Referenzfrequenz $\omega_0$ zu modulieren. In diesem Zuge
werden das Nutzsignal $U_ {sig}$ und das Referenzsignal $U_{ref}$ in einem Mischer
miteinander multipliziert\footnote{Näheres zum Aufbau in Kapitel \ref{sec:3.1}}
und anschließend als Mischsignal $U_{sig} \times U_{ref}$ über mehrere Perioden
der Modulationsfrequenz integriert, sodass sich die unerwünschten Rauschbeiträge
weitgehend herausmitteln und am Ausgang eine Gleichspannung $U_{out} \propto U_{sig} \,
\symup{cos}(\phi)$ zu messen ist, die sich proportional zu $U_{sig}$ verhält. Dabei
ist $\phi$ die veränderliche Phasenlage des Referenzsignals, die mit dem Nutzsignal synchronisiert
wird.

\begin{figure}
  \centering
  \includegraphics[scale=0.4]{theorie.png}
  \caption{Spannungsverläufe für eine sinusförmige Nutzspannung.}
  \label{fig:1}
\end{figure}
In Abbildung \ref{fig:1} wird der Signalverlauf für eine sinusförmiges Nutzspannung
\begin{equation*}
    U_{sig} = U_0 \, \symup{sin}(\omega \, t)
\end{equation*}
dargestellt. $U_{ref}$ ist hierbei durch eine Rechteckspannung mit gleicher Frequenz
moduliert. Wenn man Diese nun durch eine Fourierreihe nähert, enthält das Produkt
aus Nutz- und Referenzsignal $U_{sig} \times U_{ref}$ also nur die geraden Oberwellen
der Grundfrequenz $\omega$. % Hier eventuell die Gleichungen einfügen?
Der nachgeschaltete Tiefpassfilter unterdrückt nun diese Oberwellen. Damit erhält
man eine Gleichspannung der Form
\begin{equation}
    U_{out} = \frac{2}{\pi} \, U_0
    \label{eqn:4}
\end{equation}
Mit einer Phasendifferenz $\phi$ zwischen Nutz- und Referenzspannung wird
\eqref{eqn:4} zu
\begin{equation}
  U_{out} = \frac{2}{\pi} \, U_0 \, \symup{cos}(\phi) \ .
  \label{eqn:5}
\end{equation}
Die Ausgangsspannung wird also maximal für $\phi = 0$, also wenn es keinen Phasenunterschied
zwischen Nutz- und Referenzspannung gibt.

\section{Durchführung}
\subsection{Versuchsaufbau}
\label{sec:3.1}
\begin{figure}
  \centering
  \includegraphics[scale=0.4]{aufbau.png}
  \caption{Schematischer Aufbau eines Lock-In-Verstärkers.}
  \label{fig:2}
\end{figure}
Wie in Abbildung \ref{fig:2} zu sehen, wird $U_{sig}$ mithilfe eines Bandpasses
von Rauschanteilen höherer und niedriger Frequenzen gereinigt. Als nächstes kommt
der bereits erwähnte Mischer\footnote{siehe Kapitel \ref{sec:2}}, der Nutz- und
Referenzsignal miteinander multipliziert, wobei sich die Phase des Referenzsignals
mit dem Phasenschieber einstellen lässt. Nachgeschaltet dazu gibt es noch einen
Tiefpass, dessen Funktion bereits in Kapitel \ref{sec:2} erläutert wurde. Zum
Schluss erhält man das Ausgangssignal $U_ {out}$.

\subsection{Versuchsdurchführung}
\begin{figure}
  \centering
  \includegraphics[scale=0.3]{durch.png}
  \caption{Schaltplan eines Lock-In-Verstärkers.}
  \label{fig:3}
\end{figure}
Als erstes wurde festgestellt, welcher Ausgang der Funktionsgenerators in Abbildung
\ref{fig:3} eine variable und welcher eine konstante Spannung liefert, und welchen
Wert Diese hat. Danach wurde der Schaltplan Schritt für Schritt aufgebaut und dabei
auf einem Speicher-Oszilloskop die Signalformen überprüft und mithilfe der Print-Funktion
Skizzen des Signals nach jedem Schritt gemacht. Dabei wird der Noise-Generator
zwar eingebunden, aber noch nicht eingeschaltet, um zuerst Messungen ohne Rauschsignal
aufzunehmen. Als nächstes wird ein sinusförmiges Signal $U_{sig}$ von % Hier brauche ich die Werte aus dem Buch
$\SI{1}{\kilo\hertz}$ und $\SI{0.01}{\volt}$ erzeugt und mit einem Referenzsignal
$U_{ref}$, welches auch sinusförmig ist und die gleiche Frequenz hat, gemischt.
Nach der Integration über den Tiefpass wird die Ausgangsspannung $U_{out}$ % Hier auch die richtige Zahl
zehnmal in Abhängigkeit von der Phasenverschiebung bestimmt.

Anschließend wird der Noise-Generator eingeschaltet und alle Messungen wiederholt.
Dabei liegt die Größenordnung des Rauschsignals ungefähr in der Größenordnung
von $U_{sig}$.

\begin{figure}
  \centering
  \includegraphics[scale=0.3]{durch2.png}
  \caption{Schaltplan eines Lock-In-Verstärkers mit Photo-Detektor.}
  \label{fig:4}
\end{figure}
Als letztes wird ein Photo-Detektor in den Schaltplan eingefügt\footnote{siehe Abbildung \ref{fig:4}},
dessen LED mit $\SI{200}{\hertz}$ % Hier auch richtiger Wert
blinkt und mit einer Rechteckspannung moduliert wird. Dann wird die Lichtintensität
als Funktion des Abstandes zwischen LED und Photodiode gemessen und der maximale
Abstand $r_{max}$ bestimmt, bei dem das Licht der LED die Photodiode nachweislich
noch erreicht.

\section{Fehlerrechnung}
Es gibt:
\begin{equation}
  \bar{T} = \frac{1}{n} \sum_{i=1}^{n} T_{i}
  \label{eqn:1}
\end{equation}
den Mittelwert und:
\begin{equation}
  \sigma_{\bar{T}} = \sqrt{\frac{1}{n(n-1)} \sum_{i=1}^{n}(\bar{T}-T_i)^2}
  \label{eqn:2}
\end{equation}
den Fehler des Mittelwertes. Falls zwei fehlerbehaftete Größen in einer Gleichung
zur Bestimmung einer anderen Größe Verwendung finden, dann berechnte sich der Gesamtfehler
nach der Gaußschen Fehlerfortpflanzung zu
\begin{equation}
    \symup \Delta f(x_1, x_2, ..., x_n) = \sqrt{\left(\frac{\symup df}{\symup dx_1} \symup \Delta
    x_1 \right)^2 +    \left(\frac{\symup df}{\symup dx_2} \symup \Delta
    x_2 \right)^2 + ... + \left(\frac{\symup df}{\symup dx_n} \symup \Delta x_n \right)^2} \ .
    \label{eqn:3}
\end{equation}

\section{Auswertung}
\subsection{Verwendung des Lock-In Verstärkers mit und ohne künstlichem Rauschen}
\begin{table}
  \centering
  \caption{Mit Gain von 5 gemessene Werte}
  \label{tab:1}
  \begin{tabular}{c c c}
    \toprule
    Phase/$\si{\pi}$ & $U_{unverauscht}/\si{\volt}$ & $U_{verauscht}/\si{\volt}$ \\
    \midrule
    0.00 & 6.00 & 6.00 \\
    0.17 & 5.00 & 5.00 \\
    0.25 & 3.50 & 3.50 \\
    0.33 & 1.50 & 1.50 \\
    0.50 & -1.00 & -1.00 \\
    0.67 & -3.50 & -3.50 \\
    0.75 & -5.00 & -5.00 \\
    0.83 & -6.00 & -6.00 \\
    1.00 & -6.00 & -6.00 \\
    1.25 & -3.50 & -3.50 \\
    1.50 & 1.00 & 1.00 \\
    1.83 & 6.00 & 6.00 \\
    \bottomrule
  \end{tabular}
\end{table}
\begin{figure}
  \centering
     \includegraphics[scale=0.6]{UvonPhi.pdf}
  \caption{Auftragen der gemessenen Spannungen gegen die Phasen mit Regression. Die Spannungen sind vom Verstärkungsfaktor bereinigt.}
  \label{plot:1}
\end{figure}
Die am Tiefpassfilter mit und ohne Rauschen gemessenen Spannungen sind in Tabelle \ref{tab:1}
dargestellt. Es zeigt sich kein messbarer Unterschied zwischen unverrauschter und verrauschter Messung,
woraus auf die Ordnungsgemäße Funktionsweise des Aufbaus geschlossen werden kann.
Die Messwerte wurden bei einem Verstärkungsfaktor von 5 gemessen. Wird die Phase gegen die Spannung
aufgetragen, ergibt sich der in Grafik \ref{plot:1} dargestellte Verlauf. Werden die Messwerte mit einer
Funktion:
\begin{equation*}
  U_{out}(\phi) = A_0 \cdot \symup{cos}(\phi + \symup{\delta}\phi)
\end{equation*}
gefittet, wobei $A_0$ die Amplitude und $\symup{\delta}\phi$ die interne Phase des Aufbaus darstellt,
ergeben sich folgende Werte für Amplitude und Phase:
\begin{equation*}
  \begin{split}
    A_0 = \SI{1.228(22)}{\volt} \\
    \symup{\delta}\phi = \SI{0.177(19)}{\pi}
  \end{split}
\end{equation*}
Die gemessenen Werte folgen also dem nach \eqref{eqn:5} zu erwartenen Verlauf.

\section{Diskussion}
\newpage
\nocite{*}
\printbibliography
