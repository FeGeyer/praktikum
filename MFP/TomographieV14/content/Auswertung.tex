\section{Auswertung}
\subsection{Fehlerrechnung}
  Für die Auswertung wird als Punktschätzer der arithmetischen Mittelwert
  \begin{equation}
    \overline{T}_{\symup{arith.}} = \frac{1}{n} \sum_{i=1}^{n} T_{i}
    \label{arith}
  \end{equation}
  genutzt.
    Für die Fehlerrechnung sowie den mathematischen Teil der Auswertung wird auf $\textsc{Python}$ \cite{python}
    zurückgegriffen:\\
    Arithmetische Mittelwerte werden durch die Funktion $\textsc{mean}$ aus dem Paket $\textsc{Numpy}$ \cite{numpy}
    nach \eqref{arith} berechnet.
    Grafiken wurden mit $\textsc{matplotlib}$ \cite{matplotlib}
    erstellt.

\subsection{Quellmessung}
Zu Beginn wird eine Messung der Quelle durchgeführt, um einen Ausdruck für $I_0$
zu erhalten. Das Energiespektrum ist in Abbildung \ref{fig:1} zu sehen. Der Photopeak
liegt bei \SI{662}{\kilo\eV} und weist 10762 Ereignisse in \SI{300}{\second} auf.

\begin{figure}
  \centering
  \includegraphics[scale=0.5]{Auswertung/Plots/Quellmessung/quellmessung.pdf}
  \caption{Quellmessung der \ce{^{137}_{55}Cs}-Quelle. Dargestellt ist die Einhüllende.}
  \label{fig:1}
\end{figure}

Damit ergibt sich
\begin{equation}
  I_0 = \SI{35,873(2)}{\becquerel}
  \label{eq:i_0}
\end{equation}
als Eingangsrate.

\subsection{Messung des ersten Würfels}
Für die Messung des ersten Würfels, der nur aus der leeren Aluminiumhülle besteht,
wurde wurden die Projektionen aus Abbildung \ref{abb:2} verwendet. Die Absorptionskoeffizienten
berechnen sich für den Fall ungleicher Varianzen nach
\begin{equation}
  \vec{\mu} = \left(\textbf{A}^{\symup T} \, \textbf{W} \, \textbf{A} \right)^{-1} \cdot
  \left(\textbf{A}^{\symup T} \, \textbf{W} \, \vec{I} \right) \, ,
  \label{eqn:mu}
\end{equation}
mit $\vec{\mu} = (\mu_1, \mu_2, ..., \mu_9)^{\symup T}$. Die Zahl 9 kommt hierbei
von der Anzahl der Elementarwürfel in der Hülle. In Tabelle \ref{tab:Würfel_1}
sind die Absorptionskoeffizienten für die leere Aluminiumhülle zu sehen.

\begin{table}
  \centering
  \caption{Absorptionskoeffizienten des ersten Würfels}
  \label{tab:Würfel_1}
  \begin{tabular}{c c}
  \toprule
  $k$ & $\mu_k$ / \SI{e3}{\per\centi\meter} \\
  \midrule
  1 & \num{25.17(076)} \\
  2 & \num{12.60(056)} \\
  3 & \num{53.32(077)} \\
  4 & \num{38.46(057)} \\
  5 & \num{17.89(059)} \\
  6 & \num{26.04(057)} \\
  7 & \num{-8.54(076)} \\
  8 & \num{33.31(057)} \\
  9 & \num{7.24(76)} \\
  \bottomrule
  \end{tabular}
\end{table}

Da sich keine unterschiedlichen Elementarwürfel in der Hülle befinden, ist hier
die Angabe eines Mittelwertes sinnvoll. Dieser ergibt sich zu
\begin{equation*}
  \mu_{\symup{mittel}} = \SI{0.02283(22)}{\per\centi\meter} \, .
\end{equation*}

\subsection{Messung des zweiten Würfels}
Für den zweiten Würfel, dessen Aluminiumhülle mit Aluminium-Elementarwürfeln gefüllt
ist, wird ebenfalls \eqref{eqn:mu} verwendet. Es ergeben sich die Werte in Tabelle
\ref{tab:Würfel_2}.

\begin{table}
  \centering
  \caption{Absorptionskoeffizienten des zweiten Würfels}
  \label{tab:Würfel_2}
  \begin{tabular}{c c}
    \toprule
    $k$ & $\mu_k$ / \SI{e3}{\per\centi\meter} \\
    \midrule
    1 & \num{226.06(117)} \\
    2 & \num{212.26(86)} \\
    3 & \num{199.86(118)} \\
    4 & \num{247.03(086)} \\
    5 & \num{178.31(093)} \\
    6 & \num{261.77(087)} \\
    7 & \num{200.04(118)} \\
    8 & \num{259.52(087)} \\
    9 & \num{204.00(118)} \\
    \bottomrule
  \end{tabular}
\end{table}

Auch hier ist die Angabe eines Mittelwertes sinnvoll. Es ergibt sich
\begin{equation*}
  \mu_{\symup{mittel}} = \SI{0.22098(34)}{\per\centi\meter} \, .
\end{equation*}

\subsection{Messung des vierten Würfels}
Der vierte Würfel besteht aus verschiedenen Elementarwürfeln. Die verschiedenen
Absorptionskoeffizienten nach \eqref{eqn:mu} sind in Tabelle \ref{tab:Würfel_4}
neben den vermuteten Materialien und den Abweichungen zum den Literaturwerten aus
Tabelle \ref{tab:literatur}
zu finden. In diesem Fall ergibt die Angabe eines Mittelwertes keinen Sinn, da die
Aufgabe darin besteht, die verschiedenen Elemente zu identifizieren. Um die einzelnen
Elementarwürfel zu identifizieren, wurde die minimale relative Abweichung von den
Literaturwerten gewählt.

\begin{table}
  \centering
  \caption{Absorptionskoeffizienten des vierten Würfels}
  \label{tab:Würfel_4}
  \begin{tabular}{c c c c}
    \toprule
    $k$ & $\mu_k$ / \SI{e3}{\per\centi\meter} & Material & relative Abweichung / \si{\percent} \\
    \midrule
    1 & \num{294.08(087)} & Aluminium & \num{45.9(4)}\\
    2 & \num{1101.10(076)} & Blei & \num{11.9(1)} \\
    3 & \num{1199.73(122)} & Blei & \num{4.0(1)} \\
    4 & \num{369.61(070)} & Eisen & \num{36.1(1)} \\
    5 & \num{1139.78(103)} & Blei & \num{8.8(1)} \\
    6 & \num{1113.18(088)} & Blei & \num{10.9(1)} \\
    7 & \num{36.94(083)} & Delrin & \num{69.4(7)} \\
    8 & \num{1052.99(075)} & Blei & \num{15.7(1)} \\
    9 & \num{963.86(114)} & Blei & \num{22.9(1)} \\
    \bottomrule
  \end{tabular}
\end{table}

\section{Diskusion}
In Tabelle \ref{tab:literatur} sind die aus Massenschwächungskoeffizient $\sigma$
und Dichte $\rho$ errechneten Absorptionskoeffizienten $\mu$ zu finden.

\begin{table}
  \centering
  \caption{Literaturwerte für die Absoprtionskoeffizienten der verschiedenen
  möglichen Materialien \cite{absorp}}
  \label{tab:literatur}
  \begin{tabular}{c | c c c}
    \toprule
    Materialien & $\sigma$ / \si{\centi\meter\tothe{2}\per\gram} & $\rho$ / \si{\gram\per\centi\meter\tothe{3}}
    & $\mu$ / \si{\per\centi\meter} \\
    \midrule
    Aluminium & 0.075 & 2.699 & 0.202 \\
    Blei & 0.110 & 11.350 & 1.250 \\
    Eisen & 0.073 & 7.874 & 0.578 \\
    Messing & 0.075 & 8.284 & 0.622 \\
    Delrin & 0.086 & 1.405 & 0.121 \\
    \bottomrule
  \end{tabular}
\end{table}

Es lässt sich festhalten, dass die Messung der leeren Aluminiumhülle insgesamt
sinnvolle Werte ergeben hat. Der Mittelwert ist klein, was darauf hindeutet, dass die
Aluminiumhülle nur wenig Einfluss auf die Messung hat, was als positiv zu bewerten ist.
Der errechnete Mittelwert des zweiten Würfels, der aus Aluminium-Elementarwürfeln
besteht, passt zum Literaturwert aus Tabelle \ref{tab:literatur}. Er liegt zwar nicht
im Fehlerintervall, aber weicht nur um \SI{8.59(014)}{\percent} ab. Zur Messung des
vierten Würfels lässt sich sagen, dass die Ergebnisse durchwachsen zu bewerten sind.
Teilweise lässt sich mit einer relativ geringen Abweichung (z.B. für den dritten oder den
fünften Elementarwürfel) eine Zuordnung treffen, teilweise sind die Abweichungen
in der Nähe von \SI{50}{\percent}.

Mögliche Gründe für die Abweichungen sind unter anderem, dass die Projektionen
aus Abbildung \ref{abb:2} per Hand eingestellt werden mussten und somit unter
Umständen aufgrund von Ungenauigkeiten auch andere Elementarwürfeln als die
beabsichtigten im Strahlengang liegen. Weiterhin hat der Strahl eine endliche Ausdehung,
welche ebenfalls ein Grund dafür ist, dass sich zusätzlich noch andere Elementarwürfel
im Strahlengang befinden.
Mit einer maschinellen Einstellung der Projektionen könnte diese Fehlerquelle minimiert
werden. Außerdem ist das Stahlstück, welches parallel zum Strahlengang liegen soll,
auf einer Seite sehr beweglich, sodass nicht unbedingt sichergestellt ist, dass
die Würfel auf den tatsächlichen Strahlengang ausgerichtet wurden. Das dieses Stahlstück
während der Messung jedoch nicht bewegt wurde, ist dies ein systematischer Fehler,
der alle Messwerte beeinflusst. Außerdem besteht bei solchen Zählexperimenten
die Gefahr von statistischen Fluktuationen, deren Einfluss könnte durch längere Messungen
minimiert werden.
